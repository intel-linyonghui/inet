\documentclass{book}
\usepackage{a4wide}

%% possible fonts -- in order of preference
%%\usepackage{palatino}
\usepackage{bookman}
%%\usepackage{charter}
%%\usepackage{newcent}
%%\usepackage{times}
%%\usepackage{avant}
%%\usepackage{helvet}
%%\usepackage{sans}
%%\usepackage{chancery}

\usepackage[T1]{fontenc}
\usepackage[11pt]{moresize}
\usepackage{setspace}
\usepackage{ifpdf}
\usepackage{makeidx}
\usepackage{longtable}  %% page wrapping table environment
\usepackage{colortbl}   %% colors for tables
\usepackage{fancyvrb}   %% the "Verbatim" environment
\usepackage{fancyhdr}   %% custom headers and footers
\usepackage{multicol}
\usepackage{listings}   %% source code listings with syntax highlight (lstxxx commands)
\usepackage[tight]{shorttoc}   %% for generating a second table of contents, only containing chapter titles
\usepackage{bytefield}  %% for drawing protocol frames
\usepackage{paralist}   %% for compact lists
\usepackage[nottoc]{tocbibind}  %% makes Bibliography and Index show up in TOC
\settocbibname{References}

\setlength{\textwidth}{160mm}
%\setlength{\oddsidemargin}{12.5mm}
%\setlength{\evensidemargin}{12.5mm}
%\setlength{\topmargin}{0mm}
\setlength{\textheight}{220mm}
%\setlength{\parskip}{1ex}
%\setlength{\parindent}{5ex}

\renewcommand{\bottomfraction}{0.9}
\renewcommand{\topfraction}{0.9}
\renewcommand{\floatpagefraction}{0.9}

%% try to cure overfull hboxes
%% \tolerance=500

%% for navigation in dvi files, only needed by old teTeX versions
%%\usepackage{srcltx}

%% try this for spell checking: cat ess2002.tex | ispell -l -t -a | sort | uniq | more

%%
%% The following snippet changes the horizontal spacing between the number and
%% the title in the table of contents.
%%
%% http://tex.stackexchange.com/questions/33841/how-to-modify-the-space-between-the-numbers-and-text-of-sectioning-titles-in-the
%%
\makeatletter
 \renewcommand*\l@section{\@dottedtocline{1}{2em}{3em}}
 \renewcommand*\l@subsection{\@dottedtocline{2}{5em}{4em}}
\renewcommand*\l@chapter[2]{%
  \ifnum \c@tocdepth >\m@ne
    \addpenalty{-\@highpenalty}%
    \vskip 1.0em \@plus\p@
    \setlength\@tempdima{2em}%
    \begingroup
      \parindent \z@ \rightskip \@pnumwidth
      \parfillskip -\@pnumwidth
      \leavevmode \bfseries
      \advance\leftskip\@tempdima
      \hskip -\leftskip
      #1\nobreak\hfil \nobreak\hb@xt@\@pnumwidth{\hss #2}\par
      \penalty\@highpenalty
    \endgroup
  \fi}
\makeatother

%%
%% OMNeT++ logo, use as {\opp}
%%
\makeatletter
%%\DeclareRobustCommand{\omnetpp}{OM\-NeT\kern-.18em++\@}
\DeclareRobustCommand{\omnetpp}{OMNeT++\@}
\makeatother

\newcommand{\opp}{\omnetpp}

%%
%% PDF Header
%%
% note: \ifpdf now comes from the ifpdf package
%\newif\ifpdf
%\ifx\pdfoutput\undefined
%  \pdffalse
%\else
%  \pdfoutput=1
%  \pdftrue
%\fi
%% PDF-Info
\ifpdf
  \usepackage[pdftex]{graphicx}
  \usepackage[plainpages=false,linktocpage,bookmarksnumbered=true,pdftex]{hyperref}   %% automatic hyperlinking
  \pdfcompresslevel=9
  \pdfinfo{/Author (Andras Varga and others)
    /Title (INET Framework User's Guide)
    /Subject ()
    /Keywords (INET, INETMANET, OMNeT++, manual)}
\else
  \usepackage{graphicx}
  \usepackage[plainpages=false]{hyperref}   %% automatic hyperlinking
\fi

%%
%% Draft conditional to include unfinished parts
%%
\newif\ifdraft
%\draftfalse %% uncomment for final version
\drafttrue %% uncomment for draft version

%%
%% Generate Index
%%
\makeindex


%%
%% Link colors (hyperref package)
%%
\definecolor{MyDarkBlue}{rgb}{0.16,0.16,0.5}
%% XXX the next line apparently screws up all links except in TOC! they'll be colored nicely, but won't work.
%\hypersetup{
%    colorlinks=true,
%    linkcolor=MyDarkBlue,
%    anchorcolor=MyDarkBlue,
%    citecolor=MyDarkBlue,
%    filecolor=MyDarkBlue,
%    menucolor=MyDarkBlue,
%    runcolor=MyDarkBlue,
%    urlcolor=blue,
%}

%%
%% Heading and Footer
%%
\pagestyle{fancy}
\fancyhf{}
\renewcommand{\footrulewidth}{0.5pt}
\renewcommand{\chaptermark}[1]{\markboth{#1}{}}
\lhead{INET Framework User's Guide -- \leftmark}
\rfoot{\thepage}

%% this is used for chapter start pages
\fancypagestyle{plain}{
    \rfoot{\thepage}
}

%%
%% Use \begin{graybox}...\end{graybox} for notes
%%
\definecolor{MyGray}{rgb}{0.85,0.85,1.0}
\makeatletter\newenvironment{graybox}%
   {\begin{flushright}\begin{lrbox}{\@tempboxa}\begin{minipage}[r]{0.95\textwidth}}%
   {\end{minipage}\end{lrbox}\colorbox{MyGray}{\usebox{\@tempboxa}}\end{flushright}}%
\makeatother


\newenvironment{note}{\begin{graybox}\textbf{NOTE: }}{\end{graybox}}
\newenvironment{warning}{\begin{graybox}\textbf{WARNING: }}{\end{graybox}}
\newenvironment{caution}{\begin{graybox}\textbf{CAUTION: }}{\end{graybox}}
\newenvironment{rationale}{\begin{graybox}\textbf{Rationale: }}{\end{graybox}}
\newenvironment{important}{\begin{graybox}\textbf{IMPORTANT: }}{\end{graybox}}

%%
%% Set up listings package
%%
\lstloadlanguages{C++,make,perl,tcl,XML,R,Matlab}

%% See listings.pdf,pp20
\lstdefinelanguage{NED} {
    morekeywords={allowunconnected,bool,channel,channelinterface,connections,const,
                  default,double,extends,false,for,gates,if,import,index,inout,input,
                  int,like,module,moduleinterface,network,output,package,parameters,
                  property,simple,sizeof,string,submodules,this,true,types,volatile,
                  xml,xmldoc},
    sensitive=true,
    morecomment=[l]{//},
    morestring=[b]",
}
\lstdefinelanguage{MSG} {
    morekeywords={abstract,bool,char,class,cplusplus,double,enum,extends,false,
                  fields,int,long,message,namespace,noncobject,packet,properties,
                  readonly,short,string,struct,true,unsigned},
    sensitive=true,
    morecomment=[l]{//},
    morestring=[b]",
}
\lstdefinelanguage{inifile} {
    morekeywords={},
    sensitive=true,
    morecomment=[l]{\#},
    morestring=[b]",
}
\lstdefinelanguage{pseudocode} {
    morekeywords={if,then,else,otherwise,whenever,while},
    sensitive=true,
    morecomment=[l]{//},
    morestring=[b]",
    mathescape=true,
}

%% thick ruler on the left; also, designate backtick as LaTeX escape character
%% (e.g. \opp needs to be written as `\opp` inside listing blocks)
\lstset{
    escapechar=`,
    basicstyle=\ttfamily,
    showstringspaces=false,
    frame=leftline,
    framesep=10pt,
    framerule=3pt,
    xleftmargin=15pt
}

\definecolor{NEDRulerColor}{rgb}{0.5,1.0,0.5}  % pale green
\definecolor{MSGRulerColor}{rgb}{0.5,1.0,0.5}  % pale green
\definecolor{CPPRulerColor}{rgb}{0.8,0.5,0.2}  % pale orange
\definecolor{IniRulerColor}{rgb}{0.9,0.9,0.3}  % pale yellow
\definecolor{FileListingRulerColor}{rgb}{0.85,0.85,0.85}  % grey
%\definecolor{CommandLineRulerColor}{rgb}{0.9,0.9,0.2}
\definecolor{PseudoCodeRulerColor}{rgb}{0.0,1.0,1.0}  % cyan
\definecolor{XMLRulerColor}{rgb}{0.8,0.8,1.0}  % pale blue

%% See listings.pdf,pp39
\lstnewenvironment{ned}
    {\lstset{language=NED,rulecolor=\color{NEDRulerColor}}}
    {}
\lstnewenvironment{msg}
    {\lstset{language=MSG,rulecolor=\color{MSGRulerColor}}}
    {}
\lstnewenvironment{cpp}
    {\lstset{language=C++,rulecolor=\color{CPPRulerColor}}}
    {}
\lstnewenvironment{inifile}
    {\lstset{language=inifile,rulecolor=\color{IniRulerColor}}}
    {}
\lstnewenvironment{filelisting}
    {\lstset{language={},rulecolor=\color{FileListingRulerColor}}}
    {}
\lstnewenvironment{commandline}
    {\lstset{language={},framesep=11pt,framerule=1pt,xleftmargin=16pt}}
    {}
\lstnewenvironment{pseudocode}
    {\lstset{language=pseudocode,rulecolor=\color{PseudoCodeRulerColor}}}
    {}
\lstnewenvironment{XML}
    {\lstset{language=XML,rulecolor=\color{XMLRulerColor}}}
    {}

% add caption={#2} to display caption
\newcommand{\xmlsnippet}[2]{%
    \lstinputlisting[language=XML,rulecolor=\color{XMLRulerColor},linerange=<!\-\-#1\-\->-<!\-\-End\-\->,includerangemarker=false,firstnumber=0]{Snippets.xml}}
\newcommand{\cppsnippet}[2]{%
    \lstinputlisting[language=C++,rulecolor=\color{CPPRulerColor},linerange=//!#1-//!End,includerangemarker=false,firstnumber=0]{Snippets.cc}}
\newcommand{\msgsnippet}[2]{%
    \lstinputlisting[language=msg,rulecolor=\color{MSGRulerColor},linerange=//!#1-//!End,includerangemarker=false,firstnumber=0]{Snippets.msg}}
\newcommand{\nedsnippet}[2]{%
    \lstinputlisting[language=ned,rulecolor=\color{NEDRulerColor},linerange=//!#1-//!End,includerangemarker=false,firstnumber=0]{Snippets.ned}}
\newcommand{\inisnippet}[2]{%
    \lstinputlisting[language=inifile,rulecolor=\color{IniRulerColor},linerange=\#!#1-\#!End,includerangemarker=false,firstnumber=0]{Snippets.ini}}

%%
%% some customization
%%
\setlength{\parindent}{0pt}
\setlength{\parskip}{1ex}

%%
%% Shortcuts
%%
\newcommand{\appendixchapter}{\chapter} %% html converter needs to know which chapters are appendices

\newcommand{\tbf}{\textbf} %% bold faced text
\newcommand{\ttt}{\texttt} %% type writer font text

\newcommand{\tab}{\hspace*{5mm}} %% tabulator settings

\newcommand{\new}{$^{New!}$}
\newcommand{\changed}{$^{Changed!}$}

\newcommand{\program}{\textbf}

%% Colordefinition for table header rows (requires package colortbl)
\newcommand{\tabheadcol}{\rowcolor[gray]{0.8}}

%%
%% Module parameters list
%%
\newenvironment{params}{\begin{itemize}}{\end{itemize}}
\newcommand{\param}[2]{\item \fpar{#1}: #2}

%%
%% Function/Class/Macro/Variable/Program/Parameter/Define names
%%
%% Write the names in type writer font and do an index entry
%% Allows word wrap by automatic hyphenation
%%
%% Usage: \ffunc{take()}
%%    or: \ffunc[take()]{take(obj)}
%% the second form uses the bracketed word for the index entry
%%

\newcommand{\protocol}[1]{%
    {#1}}

%% NED type names
\newcommand{\nedtype}[2][\DefaultOpt]{\def\DefaultOpt{#2}%
  \index{#1}%
  \texttt{\hyphenchar\font=`\-\relax#2}}

%% MSG type names
\newcommand{\msgtype}[2][\DefaultOpt]{\def\DefaultOpt{#2}%
  \index{#1}%
  \texttt{\hyphenchar\font=`\-\relax#2}}

%% Function names
\newcommand{\ffunc}[2][\DefaultOpt]{\def\DefaultOpt{#2}%
  \index{#1}%
  \texttt{\hyphenchar\font=`\-\relax#2}}

%% Class names
\newcommand{\cppclass}[2][\DefaultOpt]{\def\DefaultOpt{#2}%
  \index{#1}%
  \texttt{\hyphenchar\font=`\-\relax#2}}

%% Macro names
\newcommand{\fmac}[2][\DefaultOpt]{\def\DefaultOpt{#2}%
  \index{#1}%
  \texttt{\hyphenchar\font=`\-\relax#2}}

%% Variable names
\newcommand{\fvar}[2][\DefaultOpt]{\def\DefaultOpt{#2}%
  \index{#1}%
  \texttt{\hyphenchar\font=`\-\relax#2}}

%% Program names
\newcommand{\fprog}[2][\DefaultOpt]{\def\DefaultOpt{#2}%
  \index{#1}%
  \texttt{\hyphenchar\font=`\-\relax#2}}

%% Parameter names
\newcommand{\fpar}[2][\DefaultOpt]{\def\DefaultOpt{#2}%
  \index{#1}%
  \texttt{\hyphenchar\font=`\-\relax#2}}

%% Defines
\newcommand{\fdef}[2][\DefaultOpt]{\def\DefaultOpt{#2}%
  \index{#1}%
  \texttt{\hyphenchar\font=`\-\relax#2}}

%% NED/MSG properties
\newcommand{\fprop}[2][\DefaultOpt]{\def\DefaultOpt{#2}%
  \index{#1}%
  \texttt{\hyphenchar\font=`\-\relax#2}}

%% Keywords (NED, MSG)
\newcommand{\fkeyword}[2][\DefaultOpt]{\def\DefaultOpt{#2}%
  \index{#1}%
  \textbf{\texttt{\hyphenchar\font=`\-\relax#2}}}

%% Configuration options
\newcommand{\fconfig}[2][\DefaultOpt]{\def\DefaultOpt{#2}%
  \index{#1}%
  \textbf{\texttt{\hyphenchar\font=`\-\relax#2}}}

%% File names
\newcommand{\ffilename}[2][\DefaultOpt]{\def\DefaultOpt{#2}%
  \index{#1}%
  \texttt{\hyphenchar\font=`\-\relax#2}}

%% Signals
\newcommand{\fsignal}[2][\DefaultOpt]{\def\DefaultOpt{#2}%
  \index{#1}%
  \texttt{\hyphenchar\font=`\-\relax#2}}

\newcommand{\fgate}[1]{\texttt{\hyphenchar\font=`\-\relax#1}}

%% do not number subsubsections
%\setcounter{secnumdepth}{4}

% limit the depth of TOC
\setcounter{tocdepth}{2}

%%
%% Start of document
%%
\begin{document}

%% set the image type preference
\DeclareGraphicsExtensions{.pdf,.png}

\pagestyle{empty}
\pagenumbering{roman}
\include{title}
\cleardoublepage

%%\setcounter{page}{1}
%\newpage
%%\pagenumbering{roman}

%% \shorttableofcontents{Chapters}{0}
%% \cleardoublepage

\tableofcontents
\cleardoublepage

\pagestyle{fancy}
\pagenumbering{arabic}

\include{ch-introduction}
\cleardoublepage

\include{ch-usage}
\cleardoublepage

\include{ch-networks}
\cleardoublepage

\include{ch-network-nodes}
\cleardoublepage

\chapter{Network Interfaces}
\label{cha:network-interfaces}

\section{Overview}

%TODO: MAC address, op mode, duplex mode, data rate, transmission power, queue limits, FCS mode

In INET simulations, network interface modules are the primary means of
communication between network nodes. They represent the required
combination of software and hardware elements from an operating system
point-of-view. 

Network interfaces are implemented with OMNeT++ compound modules that
conform to the \nedtype{INetworkInterface} module interface. 
Network interfaces can be further categorized as wired and wireless;
they conform to the \nedtype{IWiredInterface} and \nedtype{IWirelessInterface}
NED types, respectively, which are subtypes of \nedtype{INetworkInterface}.

\section{Built-in Network Interfaces}

INET provides pre-assembled network interfaces for several standard
protocols, protocol tunneling, hardware emulation, etc. The following list
gives the most commonly used network interfaces.

\begin{itemize}
    \item \nedtype{EthernetInterface} represents an \protocol{Ethernet} interface
    \item \nedtype{PppInterface} is for wired links using \protocol{PPP}
    \item \nedtype{Ieee80211Interface} represents a Wifi (\protocol{IEEE 802.11}) interface
    \item \nedtype{Ieee802154Interface} represents a \protocol{IEEE 802.15.4} interface
    \item \nedtype{BMacInterface}, \nedtype{LMacInterface}, \nedtype{XMacInterface} provide 
      low-power wireless sensor MAC protocols along with a simple hypothetical PHY protocol
    \item \nedtype{TunInterface} is a tunneling interface that can be directly used by applications
    \item \nedtype{LoopbackInterface} provides local loopback within the network node
    \item \nedtype{ExtInterface} represents a real-world interface, suitable for hardware-in-the-loop simulations
\end{itemize}

\section{Anatomy of Network Interfaces}

Network interfaces in the INET Framework are OMNeT++ compound modules that
contain many more components than just the corresponding layer 2 protocol
implementation. Most of these components are optional, i.e. absent by default,
and can be added via configuration.

Typical ingredients are:

\begin{itemize}
    \item \emph{Layer 2 protocol implementation}. For some interfaces such as
      \nedtype{PppInterface} this is a single module; for others like Ethernet
      and Wifi it consists of separate modules for MAC, LLC, and possibly 
      other subcomponents. 
    \item \emph{PHY model}. Some interfaces also contain separate
      module(s) that implement the physical layer. For example, 
      \nedtype{Ieee80211Interface} contains a radio module.
    \item \emph{Output queue}. This module is optional and absent by default, 
      because most MAC protocol implementations already contain an internal queue
      which is more efficient to work with. The possibility to plug in an 
      external queue module allows one to experiment with different queueing policies
      and implement QoS, RED, etc.
    \item \emph{Traffic conditioners} allow traffic shaping and policing elements
      to be added to the interface, for example to implement a Diffserv router.
    \item \emph{Hooks} allow extra modules to be inserted in the incoming
      and outgoing paths of packets. 
\end{itemize}


\subsection{Internal vs External Output Queue}

Network interfaces usually have the external queue module defined with a
parametric type like this: 

\begin{ned}
queue: <queueType> like IOutputQueue if queueType != "";
\end{ned} 

When \fpar{queueType} is empty (this is the default), the external queue 
module is absent, and the MAC (or equivalent L2) protocol will use its 
internal queue object. Conceptually, the internal queue is of inifinite size, 
but for better diagnostics one can often specify a hard limit for the queue
length in a module parameter -- if this is exceeded, the simulation 
stops with an error.

When \fpar{queueType} is not empty, it must name a NED type that 
implements the \nedtype{IOutputQueue} interface. The external 
queue module model allows modeling a finite buffer, or implement
various queueing policies for QoS and/or RED.

The most frequently used module type for external queue is 
\nedtype{DropTailQueue}, a finite-size FIFO that drops overflowing 
packets). Other queue types that implement queueing policies can be 
created by assembling compound modules from DiffServ components 
(see chapter \ref{cha:diffserv}). An example of such compound
modules is \nedtype{DiffservQueue}.

An example ini file fragment that installs drop-tail queues of size 10
on PPP interfaces:

\begin{inifile}
**.ppp[*].queueType = "DropTailQueue"
**.ppp[*].queue.frameCapacity = 10
\end{inifile}

\subsection{Traffic Conditioners}

Many network interfaces contain optional traffic conditioner submodules
defined with parametric types, like this: 

\begin{ned}
ingressTC: <ingressTCType> like ITrafficConditioner if ingressTCType != "";
egressTC: <egressTCType> like ITrafficConditioner if egressTCType != "";
\end{ned}

Traffic conditioners allow one to implement the policing and shaping actions
of a Diffserv router. They are added to the input or output packets paths  
in the network interface. (On the output path they are added before the queue 
module.) 

Traffic conditioners must implement the \nedtype{ITrafficConditioner} module
interface. Traffic conditioners can be assembled from DiffServ components 
(see chapter \ref{cha:diffserv}). There is no preassembled traffic conditioner
in INET, but you can find some in the example simulations.

An example configuration with fictituous types:

\begin{inifile}
**.ppp[*].ingressTCType = "CustomIngressTC"
**.ppp[*].egressTCType = "CustomEgressTC"
\end{inifile}


\subsection{Hooks}

Several network interfaces allow extra modules to be inserted in the incoming
and outgoing paths of packets at the top of the netwok interface. 
Hooks are added as a submodule vector with parametric type, like this: 

\begin{ned}
outputHook[numOutputHooks]: <default("Nop")> like IHook if numOutputHooks>0;
inputHook[numInputHooks]: <default("Nop")> like IHook if numInputHooks>0;
\end{ned}

This allows any number of hook modules to be added. The hook modules 
are chained in their numeric order.

Modules inserted as hooks may act as probes (for measuring or recording
traffic) or as means of modifying or perturbing the packet flow for 
experimentation. Module types implementing the \nedtype{IHook} NED interface
include \nedtype{ThruputMeter}, \nedtype{Delayer}, \nedtype{OrdinalBasedDropper},
and \nedtype{OrdinalBasedDuplicator}. 

The following ini file fragment inserts two hook modules into the output
paths of PPP interfaces, a delayer and a throughput meter:

\begin{inifile}
**.ppp[*].numOutputHooks = 2
**.ppp[*].outputHook[0].typename = "Delayer"
**.ppp[*].outputHook[1].typename = "ThruputMeter"
**.ppp[*].outputHook[0].delay = 3ms
\end{inifile}



\section{The Interface Table}

Network nodes normally contain an \nedtype{InterfaceTable} module.
The interface table is a sort of registry of all the network interfaces
in the host. It does not send or receive messages, other modules access it
via C++ function calls. Contents of the interface table can also
be inspected e.g. in Qtenv.

Network interfaces register themselves in the interface table at the
beginning of the simulation. Registration is usually the task of the
MAC (or equivalent) module. 


\section{Wired Network Interfaces}

Wired interfaces have a pair of special purpose OMNeT++ gates which represent
the capability of having an external physical connection to another network
node (e.g. Ethernet port). In order to make wired communication work,
these gates must be connected with special connections which represent the
physical cable between the physical ports. The connections must use special
OMNeT++ channels (e.g. \nedtype{DatarateChannel}) which determine datarate
and delay parameters.

Wired network interfaces are compound modules that implement the 
\nedtype{IWiredInterface} interface. INET has the following
wired network interfaces. 

\subsection{PPP}

Network interfaces for point-to-point links (\nedtype{PppInterface}) are 
described in chapter \ref{cha:ppp}. They are typically used in routers.

\subsection{Ethernet}

Ethernet interfaces (\nedtype{EthernetInterface}), alongside with models 
of Ethernet devices such as switches and hubs, are described in chapter
\ref{cha:ethernet}.

\section{Wireless Network Interfaces}

Wireless interfaces use direct sending\footnote{OMNeT++ \ttt{sendDirect()} calls} 
for communication instead of links, so their compound modules do not have
output gates at the physical layer, only an input gate dedicated to receiving. 
Another difference from the wired case is that wireless interfaces 
require (and collaborate with) a \textit{transmission medium} module 
at the network level. The medium module represents the shared transmission 
medium (electromagnetic field or acoustic medium), is responsible for 
modeling physical effects like signal attenuation, and maintains 
connectivity information. Also, while wired interfaces can do without
explicit modeling of the physical layer, a PHY module is an indispensable
part of a wireless interface.

Wireless network interfaces are compound modules that implement the 
\nedtype{IWirelessInterface} interface. In the following sections we 
give an overview of the wireless interfaces available in INET.

\subsection{Generic Wireless Interface}

The \nedtype{WirelessInterface} compound module is a generic implementation
of \nedtype{IWirelessInterface}. In this network interface, the types of the
MAC protocol and the PHY layer (the radio) are parameters:

\begin{ned}
mac: <macType> like IMacProtocol;
radio: <radioType> like IRadio if radioType != "";
\end{ned}

There are specialized versions of \nedtype{WirelessInterface} where 
the MAC and the radio modules are fixed to a particular value. 
One example is \nedtype{BMacInterface}, which contains a \nedtype{BMac}
and an \nedtype{ApskRadio}.

\subsection{IEEE 802.11}

IEEE 802.11 or Wifi network interfaces (\nedtype{Ieee80211Interface}),
alongside with models of devices acting as access points (AP),
are covered in chapter \ref{cha:80211}.

\subsection{IEEE 802.15.4}

\nedtype{Ieee802154Interface} is covered in a separate chapter, see \ref{cha:802154}.

\subsection{Wireless Sensor Networks}

MAC protocols for wireless sensor networks (WSNs) and the corresponding
network interfaces are covered in chapter \ref{cha:sensor-macs}.

\subsection{CSMA/CA} 

\nedtype{CsmaCaMac} implements an imaginary CSMA/CA-based MAC protocol with
optional acknowledgements and a retry mechanism. With the appropriate settings,
it can approximate basic 802.11b ad-hoc mode operation.

\nedtype{CsmaCaMac} provides a lot of room for experimentation: 
acknowledgements can be turned on/off, and operation parameters like
inter-frame gap sizes, backoff behaviour (slot time, minimum and maximum 
number of slots), maximum retry count, header and ACK frame sizes, bit rate,
etc. can be configured via NED parameters.

\nedtype{CsmaCaInterface} interface is a \nedtype{WirelessInterface} with
the MAC type set to \nedtype{CsmaCaMac}. 

\subsection{Acking MAC}

Not every simulation requires a detailed simulation of the lower layers.
\nedtype{AckingWirelessInterface} is a highly abstracted wireless interface 
that offers simplicity for scenarios where Layer 1 and 2 effects can be 
completely ignored, for example testing the basic functionality of a 
wireless ad-hoc routing protocol.

\nedtype{AckingWirelessInterface} is a \nedtype{WirelessInterface} 
parameterized to contain a unit disk radio (\nedtype{UnitDiskRadio})
and a trivial MAC protocol (\nedtype{AckingMac}). 

The most important parameter \nedtype{UnitDiskRadio} accepts is the 
transmission range. When a radio transmits a frame, all other radios 
within transmission range are able to receive the frame correctly, 
and radios that are out of range will not be affected at all. 
Interference modeling (collisions) is optional, and it is recommended
to turn off with \nedtype{AckingMac}.

\nedtype{AckingMac} implements a trivial MAC protocol that has packet
encapsulation and decapsulation, but no real medium access protocol. 
Frames are simply transmitted on the wireless channel as soon as the
transmitter becomes idle. (There is no carrier sense, collision avoidance, 
or collison detection.) \nedtype{AckingMac} also provides an optional 
out-of-band acknowledgement mechanism (using C++ function calls, 
not actual wirelessly sent frames), which is turned on by default.
There is no retransmission: if an acknowledgement does not arrive
after the first transmission of the packet, the MAC discards the
packet and counts it as ``given up''. 

\subsection{Shortcut}

\nedtype{ShortcutMac} implements error-free ``teleportation'' of packets 
to the peer MAC entity, with some delay computed from a transmission 
duration and a propagation delay. The physical layer is completely bypassed.
The corresponding network interface type, \nedtype{ShortcutInterface},
does not even have a radio model.

\nedtype{ShortcutInterface} is useful for modeling wireless networks
where full connectivity is assumed, and Layer 1 and Layer 2 effects
can be completely ignored. 

\section{Special-Purpose Network Interfaces}

 
\subsection{Tunnelling}

\nedtype{TunInterface} is a virtual network interface that can be used 
for creating tunnels, but it is more powerful than that.
It lets an application-layer module capture packets sent to 
the TUN interface and do whatever it pleases with it (including
sending it to a peer entity in an UDP or plain IPv4 packet.)

To set up a tunnel, add an instance of \nedtype{TunnelApp} to 
the node, and specify the protocol (IPv4 or UDP) and the remote
endpoint of the tunnel (address and port) in parameters. 

TODO example: see examples/inet/tunnel

\subsection{Local Loopback}

\nedtype{LoopbackInterface} provides local loopback within the network node.

\subsection{External Interface}

\nedtype{ExtInterface} represents a real-world interface, suitable for 
hardware-in-the-loop simulations. External interfaces are explained in 
chapter \ref{cha:emulation}.

\section{Custom Network Interfaces}

It's also possible to build custom network interfaces, the following
example shows how to build a custom wireless interface.

\nedsnippet{WirelessInterfaceExample}{Wireless interface example}

The above network interface contains very simple hypothetical MAC and PHY
protocols. The MAC protocol only provides acknowledgment without other
services (e.g., carrier sense, collision avoidance, collision detection),
the PHY protocol uses one of the predefined APSK modulations for the whole
signal (preamble, header, and data) without other services (e.g.,
scrambling, interleaving, forward error correction).


%%% Local Variables:
%%% mode: latex
%%% TeX-master: "usman"
%%% End:



\cleardoublepage

\include{ch-apps}
\cleardoublepage

\include{ch-transport}
\cleardoublepage

\include{ch-ipv4}
\cleardoublepage

\include{ch-ipv6}
\cleardoublepage

\chapter{Other Network Protocols}
\label{cha:other-network-protocols}

TODO how to choose addressing scheme?  every addressing scheme works with every protocol?  

Ipv4NetworkLayer
Ipv6NetworkLayer
WiseRouteNetworkLayer = WiseRoute + GenericArp

SimpleNetworkLayer like INetworkLayer

contains:
  np: <networkProtocolType> like INetworkProtocol,
  echo: EchoProtocol

generic: <networkLayerType> like INetworkLayer if networkLayerType != ""

set:

**.hasIpv4 = false
**.hasIpv6 = false
**.hasGn = true  <================= unused in NetworkLayerNodeBase, but referenced in TransportLayerNodeBase
**.networkLayerType = ``WiseRouteNetworkLayer''

see networklayer.ini

**.networkConfiguratorType = "Ipv4NetworkConfigurator"   ???

\section{GenericArp, GlobalArp}

TODO

\section{Flood}

\nedtype{Flood} is a simple flooding protocol for network-level broadcast.
It remembers already broadcasted messages, and does not rebroadcast 
them if it gets another copy of that message.

TODO like INetworkProtocol

\section{ProbabilisticBroadcast}

\nedtype{ProbabilisticBroadcast} is a multi-hop ad-hoc data dissemination 
protocol based on probabilistic broadcast.

This method reduces the number of packets sent on the channel (reducing the
broadcast storm problem) at the risk of some nodes not receiving the data.
It is particularly interesting for mobile networks.

The transmission probability for each attempt, the time between two transmission
attempts, the maximum number of broadcast transmissions of a packet, and
some other settings are parameters.

TODO like INetworkProtocol

\nedtype{AdaptiveProbabilisticBroadcast} is a version that automatically 
adapts transmission probabilities depending on the estimated number of 
neighbours.

\section{WiseRoute}

\nedtype{Wiseroute} is a simple loop-free routing algorithm that
builds a routing tree from a central network point, designed
for sensor networks and convergecast traffic.

The sink (the device at the center of the network) broadcasts
a route building message. Each network node that receives it
selects the sink as parent in the routing tree, and rebroadcasts
the route building message. This procedure maximizes the probability
that all network nodes can join the network, and avoids loops.
Parameter sinkAddress gives the sink network address,
rssiThreshold is a threshold to avoid using bad links (with too low
RSSI values) for routing, and routeFloodsInterval should be set to
zero for all nodes except the sink. Each routeFloodsInterval, the
sink restarts the tree building procedure. Set it to a large value
if you do not want the tree to be rebuilt.

TODO like INetworkProtocol

WiseRouteNetworkLayer

\section{GenericNetworkProtocol}

This module is a simplified generic network protocol that routes
generic datagrams using different kind of network addresses. 

TODO GenericNetworkLayer like INetworkLayer  is a compound module

routingTable: GenericRoutingTable,
gnp: GenericNetworkProtocol,
echo: EchoProtocol,
arp: GenericArp


\section{/////////////////////////////////////////////////////////}


\section{InternetCloud}

This module is an IPv4 router that can delay or drop packets (while retaining
their order) based on which interface card the packet arrived on and 
on which interface It is leaving the cloud. The delayer module is replacable.

By default the delayer module is ~MatrixCloudDelayer which lets you configure
the delay, drop and datarate parameters in an XML file. Packet flows, as defined
by incoming and outgoing interface pairs, are independent of each other.

The ~InternetCloud module can be used only to model the delay between two hops, but
it is possible to build more complex networks using several ~InternetCloud modules.

\section{PIM}

Protocol Independent Multicast -- not a network protocol 

models: \nedtype{PimSm}, \nedtype{PimDm}; \nedtype{Pim} is a compound module



%%% Local Variables:
%%% mode: latex
%%% TeX-master: "usman"
%%% End:


\cleardoublepage

\include{ch-routing}
\cleardoublepage

\include{ch-adhoc-routing}
\cleardoublepage

\include{ch-diffserv}
\cleardoublepage

\chapter{The MPLS Models}
\label{cha:mpls}

% HINT: A good MPLS primer (a 64-slide presentation):
% "MPLS for Dummies", Richard A Steenbergen <ras@nlayer.net>, nLayer Communications, Inc.
% https://www.nanog.org/meetings/nanog49/presentations/Sunday/mpls-nanog49.pdf

TODO rename Rsvp to RsvpTe
TODO rename LdpRouter to LdpMplsRouter (or LdpLsr?), and RsvpRouter to RsvpMplsRouter

\section{Overview}

TODO what is MPLS

INET provides the following components for building MPLS simulations.

The core modules are:

\begin{itemize}
  \item \nedtype{Mpls}, 
  \item \nedtype{LibTable}, 
  \item \nedtype{Ldp}, 
  \item \nedtype{Rsvp}, 
  \item \nedtype{Ted}, 
  \item \nedtype{LinkStateRouting} 
\end{itemize}

MPLS-enabled routers are:

\begin{itemize}
  \item \nedtype{LdpRouter} is an MPLS router with LDP as control protocol
  \item \nedtype{RsvpRouter} is an MPLS router with RSVP-TE as control protocol 
\end{itemize}


\section{The MPLS Module}

% collaborations with other modules, like LibTable.  libTableModule parameter

LibTable stores the LIB (Label Information Base), accessed by ~Mpls and its
associated control protocols (~Rsvp, ~Ldp) via direct C++ method calls.

xml config

TODO

\section{The LDP Module}

% collaborations with other modules

TODO

\section{The RSVP Module}

% collaborations with other modules

TODO

\section{LIB Table File Format}

The format of a LIB table file is:

The beginning of the file should begin with comments. Lines that begin with \# are treated
as comments. An empty line is required after the comments. The "LIB TABLE"
syntax must come next with an empty line. The column headers follow. This header
must be strictly "In-lbl In-intf Out-lbl Out-intf". Column
values are after that with space or tab for field separation.
The following is a sample of lib table file.

\begin{verbatim}
#lib table for MPLS network simulation test
#lib1.table for LSR1 - this is an edge router
#no incoming label for traffic from in-intf 0 &1 - LSR1 is ingress router for those traffic
#no outgoing label for traffic from in_intf 2 &3 - LSR 1 is egress router for those traffic

LIB TABLE:

In-lbl  In-intf         Out-lbl     Out-intf
1       193.233.7.90    1           193.231.7.21
2       193.243.2.1     0           193.243.2.3
\end{verbatim}


\section{The traffic.xml file}

The traffic.xml file is read by the RSVP-TE module (RSVP).
The file must be in the same folder as the executable
network simulation file.

The XML elements used in the "traffic.xml" file:

\begin{itemize}
  \item \ttt{<Traffic></Traffic>} is the root element. It may contain one or more \ttt{<Conn>} elements.
  \item \ttt{<Conn></Conn>} specifies an RSVP session. It may contain the following elements:
  \begin{itemize}
    \item \ttt{<src></src>} specifies sender IP address
    \item \ttt{<dest></dest>} specifies receiver IP address
    \item \ttt{<setupPri></setupPri>} specifies LSP setup priority
    \item \ttt{<holdingPri></holdingPri>} specifies LSP holding priority
    \item \ttt{<bandwidth></bandwidth>} specifies the requested BW.
    \item \ttt{<delay></delay>} specifies the requested delay.
    \item \ttt{<route></route>} specifies the explicit route. This is a comma-separated
      list of IP-address, hop-type pairs (also separated by comma).
      A hop type has a value of 1 if the hop is a loose hop and 0 otherwise.
  \end{itemize}
\end{itemize}

The following presents an example file:

\begin{verbatim}
<?xml version="1.0"?>
<!-- Example of traffic control file -->
<traffic>
   <conn>
       <src>10.0.0.1</src>
       <dest>10.0.1.2</dest>
       <setupPri>7</setupPri>
       <holdingPri>7</holdingPri>
       <bandwidth>400</bandwidth>
       <delay>5</delay>
   </conn>
   <conn>
       <src>11.0.0.1</src>
       <dest>11.0.1.2</dest>
       <setupPri>7</setupPri>
       <holdingPri>7</holdingPri>
       <bandwidth>100</bandwidth>
       <delay>5</delay>
   </conn>
</traffic>
\end{verbatim}

An example of using RSVP-TE as signaling protocol can be found in
ExplicitRouting folder distributed with the simulation. In this
example, a network similar to the network in LDP-MPLS example is
setup. Instead of using LDP, "signaling" parameter is set to 2 (value
of RSVP-TE handler). The following xml file is used for traffic
control. Note the explicit routes specified in the second connection.
It indicates that the route is a strict one since the values of every
hop types are 0. The route defined is 10.0.0.1 -> 1.0.0.1 ->
10.0.0.3 -> 1.0.0.4 -> 10.0.0.5 -> 10.0.1.2.

\begin{verbatim}
<?xml version="1.0"?>
<!-- Example of traffic control file -->
<traffic>
    <conn>
        <src>10.0.0.1</src>
        <dest>10.0.1.2</dest>
        <setupPri>7</setupPri>
        <holdingPri>7</holdingPri>
        <bandwidth>0</bandwidth>
        <delay>0</delay>
        <ER>false</ER>
    </conn>
    <conn>
        <src>11.0.0.1</src>
        <dest>11.0.1.2</dest>
        <setupPri>7</setupPri>
        <holdingPri>7</holdingPri>
        <bandwidth>0</bandwidth>
        <delay>0</delay>
        <ER>true</ER>
        <route>1.0.0.1,0,1.0.0.3,0,1.0.0.4,0,1.0.0.5,0,10.0.1.2,0</route>
    </conn>
</traffic>
\end{verbatim}

\section{MPLS-enabled Router Models}

MPLS-enabled routers are \nedtype{LdpRouter}, \nedtype{RsvpRouter} 

TODO 

\section{Example Network}

TODO

\section{Related Standards}

\begin{itemize}
  \item RFC 2702: Requirements for Traffic Engineering Over MPLS
  \item RFC 3031: Multiprotocol Label Switching Architecture
  \item RFC 3036: LDP Specification
  \item RFC 2205: Resource ReSerVation Protocol
  \item RFC 3209: RSVP-TE Extension to RSVP for LSP tunnels
  \item RFC 2205: RSVP Version 1 - Functional Specification
  \item RFC 2209: RSVP Message processing Version 1
\end{itemize}

%%% Local Variables:
%%% mode: latex
%%% TeX-master: "usman"
%%% End:

\cleardoublepage

\include{ch-ppp}
\cleardoublepage

\include{ch-ethernet}
\cleardoublepage

\include{ch-80211}
\cleardoublepage

\include{ch-802154}
\cleardoublepage

\include{ch-sensor-macs}
\cleardoublepage

\chapter{The Physical Layer (Transceiver Modeling)}
\label{cha:physicallayer}

\section{Overview}

Wireless network interfaces contain a radio model component, which is
responsible for modeling the physical layer (PHY).\footnote{Wired network interfaces
could similarly contain an explicit PHY model. The reason they do not is that
wired links normally have very low error rates and simple observable behavior,
and there is usually not much to be gained from modeling the physical layer in detail.}
The radio model describes the physical device that is capable of transmitting
and receiving signals on the medium. 

Conceptually, a radio model relies on several sub-models:

\begin{itemize}
  \item antenna model
  \item transmitter model 
  \item receiver model 
  \item error model (as part of the receiver model)
  \item energy consumption model 
\end{itemize}

The antenna model is shared between the transmitter model and the receiver model.
The separation of the transmitter model and the receiver model allows 
asymmetric configurations. The energy consumer model is optional, and 
it is only used when the simulation of energy consumption is necessary.

TODO multiple implementations are provided for each model. For different
level of detail (abstract/fast versus detailed), different modeling strategy, etc.

TODO explain scalar, dimensional, and ``layered''

TODO different signal representations for models of different detail levels, etc.

\section{Generic Radio}

In INET, radio models implement the \nedtype{IRadio} module interface. 
A generic, often used implementation of \nedtype{IRadio} is the 
\nedtype{Radio} NED type. \nedtype{Radio} is an active compound module, 
that is, it has an associated C++ class that encapsulates the computations.

\nedtype{Radio} contains its antenna, transmitter, receiver and energy
consumer models as submodules with parametric types:

\begin{ned}
antenna: <antennaType> like IAntenna;
transmitter: <transmitterType> like ITransmitter;
receiver: <receiverType> like IReceiver;
energyConsumer: <energyConsumerType> like IEnergyConsumer 
    if energyConsumerType != "";
\end{ned}

The following sections describe the parts of the radio model.

\section{Components of a Radio}

\subsection{Antenna Models}

The antenna model describes the effects of the physical device which converts
electric signals into radio waves, and vice versa. This model captures the
antenna characteristics that heavily affect the quality of the communication
channel. For example, various antenna shapes, antenna size and geometry, antenna
arrays, and antenna orientation causes different directional or frequency
selectivity.

The antenna model provides a position and an orientation using a mobility model
that defaults to the mobility of the node. The main purpose of this model is to
compute the antenna gain based on the specific antenna characteristics and the
direction of the signal. The signal direction is computed by the medium from the
position and the orientation of the transmitter and the receiver. The following
list provides some examples:

\begin{itemize}
  \item \nedtype{IsotropicAntenna}: antenna gain is exactly 1 in any direction
  \item \nedtype{ConstantGainAntenna}: antenna gain is a constant determined by
    a parameter
  \item \nedtype{DipoleAntenna}: antenna gain depends on the direction according
    to the dipole antenna characteristics
  \item \nedtype{InterpolatingAntenna}: antenna gain is computed by linear
    interpolation according to a table indexed by the direction angles
\end{itemize}

\subsection{Transmitter Models}

The transmitter model describes the physical process which converts packets into
electric signals. In other words, this model converts an L2 frame into a signal
that is transmitted on the medium. The conversion process and the representation
of the signal depends on the level of detail and the physical characteristics
of the implemented protocol.

There are two main levels of detail (or modeling depths):
 
\begin{itemize}
\item In the \textit{flat model}, the transmitter model skips the symbol domain 
and the sample domain representations, and it directly creates the analog domain 
representation. The bit domain representation is reduced to the bit length of 
the packet, and the actual bits are ignored.

\item In the \textit{layered model}, the conversion process involves various 
processing steps such as packet serialization, forward error correction encoding, 
scrambling, interleaving, and modulation. This transmitter model requires 
significantly more computation, but it produces accurate bit domain, 
symbol domain, and sample domain representations.
\end{itemize}

Some of the transmitter types available in INET:

\begin{itemize}
  \item \nedtype{UnitDiskTransmitter}
  \item \nedtype{ApskScalarTransmitter}
  \item \nedtype{ApskDimensionalTransmitter}
  \item \nedtype{ApskLayeredTransmitter}
  \item \nedtype{Ieee80211ScalarTransmitter}
  \item \nedtype{Ieee80211DimensionalTransmitter}
\end{itemize}


\subsection{Receiver Models}

The receiver model describes the physical process which converts electric
signals into packets. In other words, this model converts a reception, along
with an interference computed by the medium model, into a MAC packet and a
reception indication.

For a packet to be received successfully, reception must be \textit{possible}
(based on reception power, bandwidth, modulation scheme and other characteristics),
it must be \textit{attempted} (i.e. the receiver must synchronize itself on
the preamble and start receiving), and it must be \textit{successful} 
(as determined by the error model and the simulated part of the signal decoding).

In the \textit{flat model}, the receiver model skips the sample domain, the symbol domain,
and the bit domain representations, and it directly creates the packet domain
representation by copying the packet from the transmission. It uses the error
model to decide whether the reception is successful.

In the \textit{layered model}, the conversion process involves various processing steps
such as demodulation, descrambling, deinterleaving, forward error correction
decoding, and deserialization. This reception model requires much more
computation than the flat model, but it produces accurate sample domain, 
symbol domain, and bit domain representations.

Some of the receiver types available in INET:

\begin{itemize}
  \item \nedtype{UnitDiskReceiver}
  \item \nedtype{ApskScalarReceiver}
  \item \nedtype{ApskDimensionalReceiver}
  \item \nedtype{ApskLayeredReceiver}
  \item \nedtype{Ieee80211ScalarReceiver}
  \item \nedtype{Ieee80211DimensionalReceiver}
\end{itemize}


\subsection{Error Models}

Determining reception errors is a crucial part of the reception process.
There are often several different statistical error models in the literature
even for a particular physical layer. In order to support this diversity, the
error model is a separate replaceable component of the receiver. 

The error model describes how the signal to noise ratio affects the amount of
errors at the receiver. The main purpose of this model is to determine whether
the received packet has errors or not. It also computes various physical
layer indications for higher layers such as packet error rate, bit error rate,
and symbol error rate. For the layered reception model it needs to compute the
erroneous bits, symbols, or samples depending on the lowest simulated physical
domain where the real decoding starts. The error model is optional (if omitted,
all receptions are considered successful.)

The following list provides some examples:

\begin{itemize}
  \item \nedtype{StochasticErrorModel}: simplistic error model with constant
    symbol/bit/packet error rates as parameters; suitable for testing. 
  \item \nedtype{ApskErrorModel} 
  \item \nedtype{Ieee80211NistErrorModel}, \nedtype{Ieee80211YansErrorModel}, 
    \nedtype{Ieee80211BerTable\-Error\-Model}: various error models for IEEE 802.11
    network interfaces.
\end{itemize}

\subsection{Power Consumption Models}

A substantial part of the energy consumption of communication devices comes from
transmitting and receiving signals. The energy consumer model describes how the
radio consumes energy depending on its activity. This model is optional (if
omitted, energy consumption is ignored.) 

The following list provides some examples:

\begin{itemize}
  \item \nedtype{StateBasedEpEnergyConsumer}: power consumption is
    determined by the radio state (a combination of radio mode, 
    transmitter state and receiver state), and specified in 
    parameters like \fpar{receiverIdlePowerConsumption} and 
    \fpar{receiverReceivingDataPowerConsumption}, in watts.
  \item \nedtype{StateBasedCcEnergyConsumer}: similar to the previous
    one, but consumption is given in amp\`eres.
\end{itemize}

\section{Layered Radio Models}

In layered radio models, the transmitter and receiver models are split
to several stages to allow more fine-grained modeling. 

For transmission, processing steps such as packet serialization, 
forward error correction (FEC) encoding, scrambling, interleaving, and 
modulation are explicitly modeled. Reception involves the inverse
operations: demodulation, descrambling, deinterleaving, 
FEC decoding, and deserialization.

In layered radio models, these processing steps are encapsulated
in four stages, represented as four submodules in both the 
transmitter and receiver model: 

\begin{enumerate}
  \item \textit{Encoding and Decoding} describe how the packet domain 
    signal representation is converted into the bit domain, and vice versa.
  \item \textit{Modulation and Demodulation} describe how the bit domain
    signal representation is converted into the symbol domain, and vice versa.
  \item \textit{Pulse Shaping and Pulse Filtering} describe how the 
    symbol domain signal representation is converted into the sample domain, 
    and vice versa.
  \item \textit{Digital Analog and Analog Digital Conversion} describe 
    how the sample domain signal representation is converted into the 
    analog domain, and vice versa.
\end{enumerate}

In layered radio transmitters and receivers such as \nedtype{ApskLayeredTransmitter}
and \nedtype{ApskLayeredReceiver}, these submodules have parametric
types to make them replaceable. This provides immense freedom for 
experimentation.

\section{Notable Radio Models}

The \nedtype{Radio} module has several specialized versions derived
from it, where certain submodule types and parameters are set to fixed values.
This section describes some of the frequently used ones.

The radio can be replaced in wireless network interfaces by setting the
\fpar{radioType} parameter, like in the following ini file fragment.
  
\begin{inifile}
**.wlan[*].radioType = "UnitDiskRadio"
\end{inifile}

However, be aware that not all MAC protocols can be used with all radio models,
and that some radio models require a matching transmission medium module. 

\subsection{UnitDiskRadio}

\nedtype{UnitDiskRadio} provides a very simple but fast and predictable 
physical layer model. It is the implementation (with some extensions)
of the \textit{Unit Disk Graph} model, which is widely used 
for the study of wireless ad-hoc networks.
\nedtype{UnitDiskRadio} is applicable if network nodes need 
to have a finite communication range, but physical effects 
of signal propagation are to be ignored.

\nedtype{UnitDiskRadio} allows three radii to be given as parameters,
instead of the usual one: communication range, interference range, and
detection range. One can also turn off interference modeling 
(meaning that signals colliding at a receiver will all be received 
correctly), which is sometimes a useful abstraction.

\nedtype{UnitDiskRadio} needs to be used together with a special physical
medium model, \nedtype{UnitDiskRadioMedium}.

The following ini file fragment shows an example configuration.

TODO wtf about those 0 meters???

\begin{inifile}
*.radioMediumType = "UnitDiskRadioMedium"
*.host[*].wlan[*].radioType = "UnitDiskRadio"
*.host[*].wlan[*].radio.transmitter.bitrate = 2Mbps
*.host[*].wlan[*].radio.transmitter.preambleDuration = 0s
*.host[*].wlan[*].radio.transmitter.headerLength = 100b
*.host[*].wlan[*].radio.transmitter.communicationRange = 100m
*.host[*].wlan[*].radio.transmitter.interferenceRange = 0m    
*.host[*].wlan[*].radio.transmitter.detectionRange = 0m
*.host[*].wlan[*].radio.receiver.ignoreInterference = true
\end{inifile}

As a side note, if modeling full connectivity and ignoring 
interference is required, then \nedtype{ShortcutInterface} 
provides an even simpler and faster alternative.

\subsection{APSK Radio}

APSK radio models provide a hypothetical radio that simulates 
one of the well-known ASP, PSK and QAM modulations. 
(APSK stands for Amplitude and Phase-Shift Keying.)

APSK radio has scalar/dimensional, and flat/layered variants.
The flat variants, \nedtype{ApskScalarRadio} and \nedtype{ApskDimensionalRadio}
model frame transmissons in the selected modulation scheme
but without utilizing other techniques such as forward error 
correction (FEC), interleaving, spreading, etc. These radios
require matching medium models, \nedtype{ApskScalarRadioMedium}
and \nedtype{ApskDimensionalRadioMedium}.

The layered versions, \nedtype{ApskLayeredScalarRadio} 
and \nedtype{ApskLayeredDimensionalRadio} can not only
model the processing steps missing from their simpler counterparts,
they it also features configurable level of detail: the transmitter
and receiver modules have \fpar{levelOfDetail} parameters that 
control which domains are actually simulated.
These radio models must be used in conjuction with
\nedtype{ApskLayeredScalarRadioMedium} and 
\nedtype{ApskLayeredDimensionalRadioMedium}, respectively.

TODO ApskLayeredScalarRadio and ApskLayeredDimensionalRadio types are missing, actually create them!!! 

TODO limitations for usage in real-world protocol models

TODO example: 1 for flat!

TODO fragment for a layered one!


\subsection{IEEE 802.11 Radios}

TODO why 802.11 needs specialized models 

\subsection{IEEE 802.15.4 Radios}

TODO why 802.15.4 needs specialized models 

\subsection{UWB-IR Radios}

TODO what is it


%%% Local Variables:
%%% mode: latex
%%% TeX-master: "usman"
%%% End:


\cleardoublepage

\chapter{The Transmission Medium}
\label{cha:transmission-medium}

\section{Overview}

For wireless communication, an additional module is required to model the
shared physical medium where the communication takes place. This module
keeps track of transceivers, noise sources, ongoing transmissions,
background noise, and other ongoing noises.

The transmission medium is modeled as an OMNeT++ compound module with
several replaceable submodules. It contains submodules to model signal
propagation, path loss, obstacle loss, signal analog model, background
noise, and various caches for efficiency. With the help of its submodules,
the medium module computes when, where, and how signals arrive at
receivers, including the set of interfering signals and noises.

As a central component, the medium module influences performance to a large
extent, so it also provides a couple of parameters for optimization. For
example, the \nedtype{rangeFilter} parameter controls how the set of affected
receivers is determined based on their distance when the signal enters the
medium.

\section{Propagation Models}

When a transmitter starts to transmit a signal, the beginning of the signal
propagates through the transmission medium. When the transmitter ends the
transmission, the signal's end propagates similarly. The propagation model
describes how a signal moves through space over time. Its main purpose is
to compute the arrival space-time coordinates at receivers. There are two
built-in models in INET, implemented as simple modules:

\begin{itemize}
        \item \nedtype{ConstantTimePropagation} is a simplistic model where the propagation time is independent of the traveled distance. The propagation time is simply determined by a module parameter.
        \item \nedtype{ConstantSpeedPropagation} is a more realistic model where the propagation time is proportional to the traveled distance. The propagation time is independent of the transmitter and receiver movement during both signal transmission and propagation. The propagation speed is determined by a module parameter.
\end{itemize}

The default propagation model is configured as follows:

\inisnippet{PropagationModelConfigurationExample}{Propagation model configuration example}

A more accurate model could take into consideration the transmitter and
receiver movement. This effect becomes especially important for acoustic
communication, because the propagation speed of the signal is much more
comparable to the speed of the transceivers.

\section{Path Loss Models}

As a signal propagates through space its power density decreases. This is
called path loss and it is the combination of many effects such as
free-space loss, refraction, diffraction, reflection, and absorption. There
are several different path loss models in the literature, which differ in
their parameterization and application area.

In INET, a path loss model is an OMNeT++ simple module implementing a
specific path loss algorithm. Its main purpose is to compute the power loss
for a given signal, but it's also capable of estimating the range for a
given loss. The latter is useful, for example, to allow visualizing
communication range. INET contains a number of built-in path loss
algorithms, each comes with its own set of parameters:

\begin{itemize}
        \item \nedtype{FreeSpacePathLoss} models line of sight path loss for air or vacuum.
        \item \nedtype{BreakpointPathLoss} refines it using dual slope model with two separate path loss exponents.
        \item \nedtype{LogNormalShadowing} models path loss for a wide range of environments (e.g. urban areas, and buildings)
        \item \nedtype{TwoRayGroundReflection} models interference between line of sight and single ground reflection.
        \item \nedtype{TwoRayInterference} refines the above for inter-vechicle communication.
        \item \nedtype{RicianFading} is a stochastical model for the anomaly caused by partial cancellation of a signal by itself.
        \item \nedtype{RayleighFading} is a stochastical model for heavily built-up urban environments when there is no dominant propagation along the line of sight.
        \item \nedtype{NakagamiFading} further refines the above two models for cellular systems.
\end{itemize}

The following example replaces the default free-space path loss model with
log normal shadowing:

\inisnippet{PathLossConfigurationExample}{Path loss configuration example}

\section{Obstacle Loss Models}

When the signal propagates through space it also passes through physical
objects present in that space. As the signal penetrates physical objects,
its power decreases when it reflects from surfaces, and also when it’s
absorbed by their material. There are various ways to model this effect,
which differ in the trade-off between accuracy and performance.

In INET, an obstacle loss model is an OMNeT++ simple module. Its main
purpose is to compute the power loss based on the traveled path and the
signal frequency. The obstacle loss models most often use the physical
environment model to determine the set of penetrated physical objects.
INET contains a few built-in obstacle loss models:

\begin{itemize}
        \item \nedtype{IdealObstacleLoss} model determines total or no power loss at all by checking if there is any obstructing physical object along the straight propagation path.
        \item \nedtype{DielectricObstacleLoss} computes the power loss based on the accurate dielectric and reflection loss along the straight path considering the shape, position, orientation, and material of obstructing physical objects.
\end{itemize}

By default, the medium module doesn't contain any obstacle loss model, but
configuring one is very simple:

\inisnippet{ObstacleLossModelConfigurationExample}{Obstacle loss model configuration example}

Statistical obstacle loss models are also possible but currently not provided.

\section{Background Noise Models}

Thermal noise, cosmic background noise, and other random fluctuations of
the electromagnetic field affect the quality of the communication channel.
This kind of noise doesn’t come from a particular source, so it doesn’t
make sense to model its propagation through space. The background noise
model describes instead how it changes over space and time.

In INET, a background noise model is an OMNeT++ simple module. Its main
purpose is to compute the analog representation of the background noise for
a given space-time interval. For example,
\nedtype{IsotropicScalarBackgroundNoise} computes a background noise that is
independent of space-time coordinates, and its scalar power is determined
by a module parameter.

The simplest background noise model can be configured as follows:

\inisnippet{BackgroundNoiseModelConfigurationExample}{Background noise model configuration example}

\section{Analog Models}

The analog signal is a complex physical phenomenon which can be modeled in
many different ways. Choosing the right analog domain signal representation
is the most important factor in the trade-off between accuracy and
performance. The analog model of the transmission medium determines how
signals are represented while being transmitted, propagated, and received.

In INET, an analog model is an OMNeT++ simple module. Its main purpose is
to compute the received signal from the transmitted signal. The analog
model combines the effect of the antenna, path loss, and obstacle loss
models. Transceivers must be configured transmit and receive signals
according to the representation used by the analog model.

The most commonly used analog model, which uses a scalar signal power
representation over a frequency and time interval, can be condigured as
follows:

\inisnippet{AnalogModelConfigurationExample}{Analog model configuration example}

\section{Neighbor Cache}

Transceivers are considered neighbors if successful communication is
possible between them. For wired communication it’s easy to determine
which transceivers are neighbors, because they are connected by wires. In
contrast, in wireless communication determining which transceivers are
neighbors isn’t obvious at all.

In INET, a neighbor cache model is an OMNeT++ simple module which provides
an efficient way of keeping track of the neighbor relationship between
transceivers. Its main purpose is to compute the set of affected receivers
for a given transmission. All built-in models in INET provide a
conservative approximation only, because they update their state
periodically:

\begin{itemize}
        \item \nedtype{NeighborListNeighborCache} maintains a separate neighbor list for each transceiver.
        \item \nedtype{GridNeighborCache} organizes transceivers in a 3D grid with constant cell size.
        \item \nedtype{QuadTreeNeighborCache} organizes transceivers in a 2D quad tree (ignoring the Z axis) with constant node size.
\end{itemize}

\inisnippet{NeighborCacheModelConfigurationExample}{Neighbor cache model configuration example}

%%% Local Variables:
%%% mode: latex
%%% TeX-master: "usman"
%%% End:

\cleardoublepage

\chapter{The Physical Environment}
\label{cha:environment}

\section{Overview}

Wireless networks are heavily affected by the physical environment, and the
requirements for today's ubiquitous wireless communication devices are
increasingly demanding. Cellular networks serve densely populated urban
areas, wireless LANs need to be able to cover large buildings with several
offices, low-power wireless sensors must tolerate noisy industrial
environments, batteries need to remain operational under various external
conditions, and so on.

The propagation of radio signals, the movement of communicating agents,
battery exhaustion, etc., depend on the surrounding physical environment.
For example, signals can be absorbed by objects, can pass through objects,
can be refracted by surfaces, can be reflected from surfaces, or battery
nominal capacity might depend on external temperature. These effects cannot
be ignored in high-fidelity simulations.

In order to help the modeling process, the model of the physical
environment is separated from the rest of the simulation models. The main
goal of the physical environment model is to describe buildings, walls,
vegetation, terrain, weather, and other physical objects and conditions
that might have effects on radio signal propagation, movement, batteries,
etc. This separation makes the model reusable by all other simulation
models that depend on these circumstances.

The following sections provide a brief overview of the physical environment
model.

\section{The Physical Environment Model}

The physical environment is represented in an INET simulation by a
\nedtype{PhysicalEnvironment} module. This module normally has one instance
in the network, and acts as a database that other parts of the simulation
can query at runtime.

\section{Global Physical Properties}

The physical environment model stores the following global
properties:

\begin{itemize}
  \item \textit{space limits}: global bounds for the 3-dimensional space
  \item \textit{temperature}: global parameter for temperature-dependent models
\end{itemize}

Space limits are useful for limiting the propagation and reflection of
radio signals, to constrain movement of communicating agents, and for
detecting incorrectly positioned physical objects.

Temperature can be useful for modeling batteries, as it affects the
maximum capacity, internal resistance, self-discharge and other properties
of real-life electrochemical energy storage devices.

\section{Physical Objects}

The most important aspect of the physical environment is the objects which
are present in it. For example, simulating an indoor Wifi scenario may need
to model walls, floors, ceilings, doors, windows, furniture, and similar
objects, because they all affect signal propagation.

Objects are located in space, and have shapes and materials. The INET
physical layer infrastructure supports basic shapes and homogeneous
materials, which simplifies description and still allows for a reasonable
approximation of reality. Physical objects in INET have the following
properties:

\begin{itemize}
  \item \textit{shape}: describes the 3-dimensional shape of the object,
    independent of its position and orientation
  \item \textit{position}: determines where the object is located in the 3-dimensional space
  \item \textit{orientation}: determines how the object is rotated relative to its
    default orientation
  \item \textit{material}: describes material-specific physical properties
  \item \textit{graphical properties}: provides parameters for better visualization
\end{itemize}

Physical objects in INET are stationary, they cannot change their position
or orientation over time.

Since the shape of the physical objects might be quite diverse, the model
is designed to be extensible with new shapes. Concave shapes are not yet
supported, such shapes can be represented by splitting them up into smaller,
convex parts. The current implementation provides the following shapes:

\begin{itemize}
  \item \textit{sphere}: specified by a radius
  \item \textit{cuboid}: specified by a length, a width, and a height
  \item \textit{prism}: specified by a 2-dimensional polygon base and a height
  \item \textit{polyhedron}: specified by the convex hull of a set of
    3-dimensional vertices
\end{itemize}

\section{Visualization}

The \nedtype{PhysicalEnvironment} module is capable is visualizing the objects on the
user interface. Rendering makes use of the following graphical object properties:

\begin{itemize}
  \item \textit{line width}: affects surface outline
  \item \textit{line color}: affects surface outline
  \item \textit{fill color}: affects surface fill
  \item \textit{opacity}: affects surface outline and fill
  \item \textit{tags}: allows filtering objects on the graphical user interface
\end{itemize}

The projection of 3D objects to the 2D canvas can be parameterized with an
arbitrary view angle. The default view angle is the Z axis (i.e. top view).
The view angle can also be changed during runtime, by changing the
appropriate module parameter.

The projection mechanism can be accessed by other models (e.g. mobility
models) for their own visualizations.

\section{Specifying Physical Objects}

Physical objects are defined for the \nedtype{PhysicalEnvironment} module
in an XML document. The document format allows one to define physical objects
together with their properties, and one can also define shapes and materials
that are shared (i.e. referenced) by several objects.

The following example shows some shapes, materials and objects defined in XML:

\begin{verbatim}
<environment>
  <!-- shapes and materials -->
  <shape id="1" type="sphere" radius="10"/>
  <shape id="2" type="cuboid" size="20 30 40"/>
  <shape id="3" type="prism" height="100"
         points="0 0 100 0 100 100 0 100"/>
  <shape id="4" type="polyhedron"
         points="0 0 0 100 0 0 100 100 0 0 100 0 ..."/>
  <material id="1" resistivity="100"
         relativePermittivity="4.5" relativePermeability="1"/>

  <!-- an object that uses a previously defined shape and material -->
  <object position="min 100 200 0" orientation="45 -30 0"
          shape="1" material="1"
          line-color="0 0 0" fill-color="112 128 144" opacity="0.5"/>

  <!-- an object defined with an in-line shape -->
  <object position="min 100 200 0" orientation="45 -30 0"
          shape="cuboid 20 30 40" material="concrete"
          line-color="0 0 0" fill-color="112 128 144" tags="Building"/>
</environment>
\end{verbatim}

For more details, please refer to the documentation of the
\nedtype{PhysicalEnvironment} module.

\section{Data Structure}

In order to model the physical environment in detail, a scenario might contain
several thousands or even more physical objects. Simulation models might
need to query these objects quite often. For example, when the physical layer
computes obstacle loss for a transmission, it needs to find the obstructing
physical objects for each receiver. This requires computing the intersection
between physical objects and the path traveled by the radio signal.

To speed up the computation of intersections, INET stores physical objects
in a highly efficient data structure, which currently can be one of the
following:

\begin{itemize}
  \item \nedtype{GridObjectCache}: organizes objects into a 3D spatial grid with
    a configurable constant cell size, where cells contain the objects that
    intersect with them
  \item \nedtype{BvhObjectCache}: organizes objects into a 3D tree, where
    leaves contain a configurable number of closely positioned objects.
    (This data structure is similar to quadtree and octree, but is designed for
    storing finite-sized objects.)
\end{itemize}

The physical environment model uses \nedtype{GridObjectCache} by default.


\section{///////////// NEW STUFF://////////////}

\section{Ovewview}

% Why physical environments are needed?
Wireless networks are heavily affected by the physical environment.
Propagation of signals, movement of communicating agents, energy
consumption of devices, all depend on the surrounding physical environment.
For example, signals are absorbed by physical objects and reflected from
their surfaces, or battery capacity might depend on external temperature.
The main purpose of the physical environment model is to describe
buildings, walls, furniture, vegetation, terrain, weather, and other
physical objects and conditions that might have profound effects on the
simulation.

In INET, the physical environment is modeled by the
\nedtype{PhysicalEnvironment} compound module. This module normally has one
instance in the network, and it provides services for other parts of the
simulation. It contains submodules to model physical objects and the ground
as well as a few parameters for the physical properties of the environment.

\section{Physical Objects}
% Why physical objects are needed?

The most important aspect of the physical environment is the objects which
are present in it. For example, simulating an indoor Wifi scenario may need
to model walls, floors, ceilings, doors, windows, furniture, and similar
objects, because they all affect signal propagation. Objects are located in
space, and have shapes and materials. The physical environment model
supports basic shapes and homogeneous materials, which is a simplified
description but still allows for a reasonable approximation of reality.
Physical objects in INET have the following properties:

% What are the properties of physical objects?
\begin{itemize}
        \item \emph{shape} describes the object in 3D independent of its position and orientation.
        \item \emph{position} determines where the object is located in the 3D space.
        \item \emph{orientation} determines how the object is rotated relative to its default orientation.
        \item \emph{material} describes material specific physical properties.
        \item \emph{graphical properties} provide parameters for better visualization.
\end{itemize}

Physical objects in INET are stationary, they cannot change their position
or orientation over time. Since the shape of the physical objects might be
quite diverse, the model is designed to be extensible with new shapes.
INET provides the following shapes:

\begin{itemize}
        \item \emph{sphere} shapes are specified by a radius
        \item \emph{cuboid} shapes are specified by a length, a width, and a height
        \item \emph{prism} shapes are specified by a 2D polygon base and a height
        \item \emph{polyhedron} shapes are specified by the convex hull of a set of 3D vertices
\end{itemize}

The following example shows how to define various physical objects using
the XML syntax supported by the physical environment:

\xmlsnippet{DefiningPhysicalObjectsExample}{Defining physical objects example}

In order to load the above XML file, the following configuration could be
used:

\inisnippet{PhysicalObjectsConfigurationExample}{Physical objects configuration example}

\section{Ground Models}

% Why ground models are needed?
In inter-vehicle simulations the terrain has profound effects on signal
propagation. For example, vehicles on the opposite sides of a mountain
cannot directly communicate with each other.

% What is the purpose of ground models?
A ground model describes the 3D surface of the terrain. Its main purpose is
to compute a position on the surface underneath an particular position.

% What ground models are available?
INET contains the following built-in ground models implemented as
OMNeT++ simple modules:

\begin{itemize}
        \item \nedtype{FlatGround} is a trivial model which provides a flat surface parallel to the XY plane at a certain height.
        \item \nedtype{OsgEarthGround} is a more realistic model (based on \program{osgEarth}) which provides a terrain surface.
\end{itemize}

\section{Geographic Coordinate System Models}

% Why geographic coordinate systems are needed?
In order to run high fidelity simulations, it is often required to embed
the communication network into a real world map. With the new OMNeT++ 5
version, INET already provides support for 3D maps using
\program{osgEarth} for visualization and \program{openstreetmap} as the map
provider.

However, INET carries out all geometric computation internally (including
signal propagation and path loss) in a 3D Euclidean coordinate system. The
discrepancy between the internal playground coordinate system and the usual
geographic coordinate systems must be resolved.

% What is the purpose of geographic coordinate system models?
A geographic coordinate system model maps playground coordinates to
geographic coordinates, and vice versa. Such a model allows positioning
physical objects and describing network node mobility using geographical
coordinates (e.g longitude, latitude, altitude).

% What geographic coordinate system models are provided?
In INET, a geographic coordinate system model is implemented as an OMNeT++
simple module:

\begin{itemize}
        \item \nedtype{SimpleGeographicCoordinateSystem} provides a trivial linear approximation without any external dependency.
        \item \nedtype{OsgGeographicCoordinateSystem} provides an accurate mapping using the external \program{osgEarth} library.
\end{itemize}

% How geographic coordinate system modules are used?
In order to use geographic coordinates in a simulation, a geographic
coordinate system module must be included in the network. The desired
physical environment module and mobility modules must be configured (using
module path parameters) to use the geographic coordinate system module. The
following example also shows how the geographic coordinate system module
can be configured to place the playground at a particular geographic
location and orientation.

\inisnippet{GeographicCoordinateSystemConfigurationExample}{Geographic coordinate system configuration example}

\subsubsection*{Object Cache Models}

% Why object cache models are needed?
If a simulation contains a large number of physical objects, then signal
propagation may become computationally very expensive. The reason is that
the transmission medium model must check each line of sight path between
all transmitter and receiver pairs against all physical objects.

% What is the purpose of object cache models?
An object cache model organizes physical objects into a data structure
which provides efficient geometric queries. Its main purpose is to iterate
all physical objects penetrated by a 3D line segment.

% What object cache models are provided?
In INET, an object cache model is implemented as an OMNeT++ simple module:

\begin{itemize}
        \item \nedtype{GridObjectCache} organizes objects into a fixed cell size 3D spatial grid.
        \item \nedtype{BvhObjectCache} organizes objects into a tree data structure based on recursive 3D volume division.
\end{itemize}


%%% Local Variables:
%%% mode: latex
%%% TeX-master: "usman"
%%% End:


\cleardoublepage

\include{ch-mobility}
\cleardoublepage

\include{ch-power}
\cleardoublepage

\include{ch-emulation}
\cleardoublepage

\chapter{Network Autoconfiguration}
\label{cha:network-autoconfiguration}

\section{Overview}

TODO: this describes static autoconfiguration of IPv4 networks

\section{Configuring IPv4 Networks}

An IPv4 network is composed of several nodes like hosts, routers,
switches, hubs, Ethernet buses, or wireless access points.
The nodes having a IPv4 network layer (hosts and routers) should be
configured at the beginning of the simulation. The configuration
assigns IP addresses to the nodes, and fills their routing tables.
If multicast forwarding is simulated, then the multicast routing
tables also must be filled in.

% TODO define nodes, IP nodes, routers, multicast routers

The configuration can be manual (each address and route is fully specified
by the user), or automatic (addresses and routes are generated by
a configurator module at startup).

Before version 1.99.4 INET offered \nedtype{Ipv4FlatNetworkConfigurator}
for automatic and routing files for manual configuration.
Both had serious limitations, so a new configurator has been added
in version 1.99.4: \nedtype{Ipv4NetworkConfigurator}. This configurator
supports both fully manual and fully automatic configuration. It
can also be used with partially specified manual configurations,
the configurator fills in the gaps automatically.

The next section describes the usage of \nedtype{Ipv4NetworkConfigurator}.
The legacy solutions \nedtype{Ipv4FlatNetworkConfigurator} and 
routing files are described in subsequent sections.

\subsection{Ipv4NetworkConfigurator}
\label{subsec:ipv4configurator}

The \nedtype{Ipv4NetworkConfigurator} assigns IP addresses and sets up
static routing for an IPv4 network.

It assigns per-interface IP addresses, strives to take subnets into account,
and can also optimize the generated routing tables by merging routing entries.

Hierarchical routing can be set up by using only a fraction of configuration
entries compared to the number of nodes. The configurator also does
routing table optimization that significantly decreases the size of routing
tables in large networks.

The configuration is performed in stage 2 of the initialization. At this
point interface modules (e.g. PPP) has already registered their interface
in the interface table. If an interface is named \ttt{ppp[0]}, then the
corresponding interface entry is named \ttt{ppp0}. This name can be used
in the config file to refer to the interface.

The configurator goes through the following steps:

\begin{enumerate}
  \item  Builds a graph representing the network topology. The graph
     will have a vertex for every module that has a @node property (this
     includes hosts, routers, and L2 devices like switches, access points,
     Ethernet hubs, etc.) It also assigns weights to vertices and edges that
     will be used by the shortest path algorithm when setting up routes.
     Weights will be infinite for IP nodes that have IP forwarding disabled
     (to prevent routes from transiting them), and zero for all other nodes
     (routers and and L2 devices). Edge weights are chosen to be inversely
     proportional to the bitrate of the link, so that the configurator
     prefers connections with higher bandwidth. For internal purposes,
     the configurator also builds a table of all "links" (the link data
     structure consists of the set of network interfaces that are
     on the same point-to-point link or LAN)

  \item  Assigns IP addresses to all interfaces of all nodes. The
     assignment process takes into consideration the addresses and netmasks
     already present on the interfaces (possibly set in earlier initialize
     stages), and the configuration provided in the XML format (described
     below). The configuration can specify "templates" for the address
     and netmask, with parts that are fixed and parts that can be chosen
     by the configurator (e.g. "10.0.x.x"). In the most general case,
     the configurator is allowed to choose any address and netmask for all
     interfaces (which results in automatic address assignment). In the most
     constrained case, the configurator is forced to use the requested addresses
     and netmasks for all interfaces (which translates to manual address assignment).
     There are many possible configuration options between these two extremums. The
     configurator assigns addresses in a way that maximizes the number of
     nodes per subnet. Once it figures out the nodes that belong to a single
     subnet it, will optimize for allocating the longest possible netmask.
     The configurator might fail to assign netmasks and addresses according
     to the given configuration parameters; if that happens, the assignment
     process stops and an error is signalled.

  \item  Adds the manual routes that are specified in the configuration.

  \item  Adds static routes to all routing tables in the network. The
     configurator uses Dijkstra's weighted shortest path algorithm to find
     the desired routes between all possible node pairs. The resulting
     routing tables will have one entry for all destination interfaces in the
     network. The configurator can be safely instructed to add default routes
     where applicable, significantly reducing the size of the host routing
     tables. It can also add subnet routes instead of interface routes further
     reducing the size of routing tables. Turning on this option requires
     careful design to avoid having IP addresses from the same subnet on
     different links. CAVEAT: Using manual routes and static route generation
     together may have unwanted side effects, because route generation ignores
     manual routes.

  \item  Then it optimizes the routing tables for size. This optimization allows
     configuring larger networks with smaller memory footprint and makes the
     routing table lookup faster. The resulting routing table might be
     different in that it will route packets that the original routing table
     did not. Nevertheless the following invariant holds: any packet routed
     by the original routing table (has matching route) will still be routed
     the same way by the optimized routing table.

  \item  Finally it dumps the requested results of the configuration. It can
     dump network topology, assigned IP addresses, routing tables and its
     own configuration format.
\end{enumerate}

The module can dump the result of the configuration in the XML format
which it can read. This is useful to save the result of a time consuming
configuration (large network with optimized routes), and use it as
the config file of subsequent runs.

\subsubsection*{Network topology graph}

The network topology graph is constructed from the nodes
of the network. The node is a module having a @node property
(this includes hosts, routers, and L2 devices like switches,
 access points, Ethernet hubs, etc.). An IP node is a node
that contains an \nedtype{InterfaceTable} and a \nedtype{Ipv4RoutingTable}.
A router is an IP node that has multiple network interfaces,
and IP forwarding is enabled in its routing table module.
In multicast routers the \fpar{forwardMulticast} parameter
is also set to \fkeyword{true}.

A link is a set of interfaces that can send datagrams to each other
without intervening routers. Each interface belongs to exactly
one link. For example two interface connected
by a point-to-point connection forms a link. Ethernet interfaces
connected via buses, hubs or switches.
The configurator identifies links by discovering
the connections between the IP nodes, buses, hubs, and switches.

Wireless links are identified by the \fpar{ssid} or \fpar{accessPointAddress}
parameter of the 802.11 management module. Wireless interfaces
whose node does not contain a management module are supposed
to be on the same wireless link. Wireless links can also be
configured in the configuration file of \nedtype{Ipv4NetworkConfigurator}:
\begin{verbatim}
<config>
  <wireless hosts="area1.*" interfaces="wlan*">
</config>
\end{verbatim}
puts wlan interfaces of the specified hosts into the same wireless link.

If a link contains only one router, it is marked as the gateway
of the link. Each datagram whose destination is outside the link
must go through the gateway.

\subsubsection*{Address assignment}

Addresses can be set up manually by giving the address and netmask for
each IP node. If some part of the address or netmask is unspecified,
then the configurator can fill them automatically. Unspecified fields
are given as an ``x'' character in the dotted notation of the address.
For example, if the address is specified as 192.168.1.1 and the
netmask is 255.255.255.0, then the node address will be 192.168.1.1
and its subnet is 192.168.1.0. If it is given as 192.168.x.x and
255.255.x.x, then the configurator chooses a subnet address in the range
of 192.168.0.0 - 192.168.255.252, and an IP address within the chosen
subnet. (The maximum subnet mask is 255.255.255.252 allows 2 nodes in the subnet.)

The following configuration generates network addresses below the 10.0.0.0
address for each link, and assign unique IP addresses to each host:

\begin{verbatim}
<config>
  <interface hosts="*" address="10.x.x.x" netmask="255.x.x.x"/>
</config>
\end{verbatim}

The configurator tries to put nodes on the same link into the same subnet,
so its enough to configure the address of only one node on each link.

The following example configures a hierarchical network in a way that keeps
routing tables small.
\begin{verbatim}
<config>
  <interface hosts="area11.lan1.*" address="10.11.1.x" netmask="255.255.255.x"/>
  <interface hosts="area11.lan2.*" address="10.11.2.x" netmask="255.255.255.x"/>
  <interface hosts="area12.lan1.*" address="10.12.1.x" netmask="255.255.255.x"/>
  <interface hosts="area12.lan2.*" address="10.12.2.x" netmask="255.255.255.x"/>
  <interface hosts="area*.router*" address="10.x.x.x" netmask="x.x.x.x"/>
  <interface hosts="*" address="10.x.x.x" netmask="255.x.x.0"/>
</config>
\end{verbatim}

The XML configuration must contain exactly one \verb!<config>! element. Under the
root element there can be multiple of the following elements:

The interface element provides configuration parameters for one or more
interfaces in the network. The selector attributes limit the scope where
the interface element has effects. The parameter attributes limit the
range of assignable addresses and netmasks.
The \verb!<interface>! element may contain the following attributes:
\begin{compactitem}
    \item \ttt{@hosts}
      Optional selector attribute that specifies a list of host name patterns.
      Only interfaces in the specified hosts are affected. The pattern might
      be a full path starting from the network, or a module name anywhere in
      the hierarchy, and other patterns similar to ini file keys. The default
      value is "*" that matches all hosts.
      e.g. "subnet.client*" or "host* router[0..3]" or "area*.*.host[0]"

    \item \ttt{@names}
      Optional selector attribute that specifies a list of interface name
      patterns. Only interfaces with the specified names are affected. The
      default value is "*" that matches all interfaces.
      e.g. "eth* ppp0" or "*"

    \item \ttt{@towards}
      Optional selector attribute that specifies a list of host name patterns.
      Only interfaces connected towards the specified hosts are affected. The
      specified name will be matched against the names of hosts that are on
      the same LAN with the one that is being configured. This works even if
      there's a switch between the configured host and the one specified here.
      For wired networks it might be easier to specify this parameter instead
      of specifying the interface names. The default value is "*".
      e.g. "ap" or "server" or "client*"

    \item \ttt{@among}
      Optional selector attribute that specifies a list of host name patterns.
      Only interfaces in the specified hosts connected towards the specified
      hosts are affected.
      The 'among="X Y Z"' is same as 'hosts="X Y Z" towards="X Y Z"'.

    \item \ttt{@address}
      Optional parameter attribute that limits the range of assignable
      addresses. Wildcards are allowed with using 'x' as part of the address
      in place of a byte. Unspecified parts will be filled automatically by
      the configurator. The default value "" means that the address will not
      be configured. Unconfigured interfaces still have allocated addresses
      in their subnets allowing them to become configured later very easily.
      e.g. "192.168.1.1" or "10.0.x.x"

    \item \ttt{@netmask}
      Optional parameter attribute that limits the range of assignable
      netmasks. Wildcards are allowed with using 'x' as part of the netmask
      in place of a byte. Unspecified parts will be filled automatically be
      the configurator. The default value "" means that any netmask can be
      configured.
      e.g. "255.255.255.0" or "255.255.x.x" or "255.255.x.0"

    \item \ttt{@mtu}                number
      Optional parameter attribute to set the MTU parameter in the interface.
      When unspecified the interface parameter is left unchanged.

    \item \ttt{@metric}                number
      Optional parameter attribute to set the Metric parameter in the interface.
      When unspecified the interface parameter is left unchanged.
\end{compactitem}

Wireless interfaces can similarly be configured by adding
\verb!<wireless>! elements to the configuration. Each \verb!<wireless>!
element with a different id defines a separate subnet.
\begin{compactitem}
    \item \ttt{@id} (optional)
      identifies wireless network, unique value used if missed

    \item \ttt{@hosts}
      Optional selector attribute that specifies a list of host name patterns.
      Only interfaces in the specified hosts are affected. The default value
      is "*" that matches all hosts.

    \item \ttt{@interfaces}
      Optional selector attribute that specifies a list of interface name
      patterns. Only interfaces with the specified names are affected. The
      default value is "*" that matches all interfaces.
\end{compactitem}


\subsubsection{Multicast groups}

Multicast groups can be configured by adding \verb!<multicast-group>!
elements to the configuration file. Interfaces belongs to a multicast
group will join to the group automatically.

For example
\begin{verbatim}
<config>
  <multicast-group hosts="router*" interfaces="eth*" address="224.0.0.5"/>
</config>
\end{verbatim}
adds all Ethernet interfaces of nodes whose name starts with ``router''
to the 224.0.0.5 multicast group.

The \verb!<multicast-group>! element has the following attributes:
\begin{compactitem}
    \item \ttt{@hosts}
      Optional selector attribute that specifies a list of host name patterns.
      Only interfaces in the specified hosts are affected. The default value
      is "*" that matches all hosts.

    \item \ttt{@interfaces}
      Optional selector attribute that specifies a list of interface name
      patterns. Only interfaces with the specified names are affected. The
      default value is "*" that matches all interfaces.

    \item \ttt{@towards}
      Optional selector attribute that specifies a list of host name patterns.
      Only interfaces connected towards the specified hosts are affected.
      The default value is "*".

    \item \ttt{@among}
      Optional selector attribute that specifies a list of host name patterns.
      Only interfaces in the specified hosts connected towards the specified
      hosts are affected.
      The 'among="X Y Z"' is same as 'hosts="X Y Z" towards="X Y Z"'.

    \item \ttt{@address}
      Mandatory parameter attribute that specifies a list of multicast group
      addresses to be assigned. Values must be selected from the valid range
      of multicast addresses.
      e.g. "224.0.0.1 224.0.1.33"
\end{compactitem}


\subsubsection*{Manual route configuration}

The \nedtype{Ipv4NetworkConfigurator} module allows the user
to fully specify the routing tables of IP nodes at the beginning
of the simulation.

The \verb!<route>! elements of the configuration add a route to the
routing tables of selected nodes. The element has the following attributes:
\begin{compactitem}
    \item \ttt{@hosts}
      Optional selector attribute that specifies a list of host name patterns.
      Only routing tables in the specified hosts are affected. The default
      value "" means all hosts will be affected.
      e.g. "host* router[0..3]"

    \item \ttt{@destination}
      Optional parameter attribute that specifies the destination address in
      the route (L3AddressResolver syntax). The default value is "*".
      e.g. "192.168.1.1" or "subnet.client[3]" or "subnet.server(ipv4)" or "*"

    \item \ttt{@netmask}
      Optional parameter attribute that specifies the netmask in the route.
      The default value is "*".
      e.g. "255.255.255.0" or "/29" or "*"

    \item \ttt{@gateway}
      Optional parameter attribute that specifies the gateway (next-hop)
      address in the route (L3AddressResolver syntax). When unspecified
      the interface parameter must be specified. The default value is "*".
      e.g. "192.168.1.254" or "subnet.router" or "*"

    \item \ttt{@interface}
      Optional parameter attribute that specifies the output interface name
      in the route. When unspecified the gateway parameter must be specified.
      This parameter has no default value.
      e.g. "eth0"

    \item \ttt{@metric}
      Optional parameter attribute that specifies the metric in the route.
      The default value is 0.
\end{compactitem}

Multicast routing tables can similarly be configured by adding
\verb!<multicast-route>! elements to the configuration.
\begin{compactitem}
    \item \ttt{@hosts}
      Optional selector attribute that specifies a list of host name patterns.
      Only routing tables in the specified hosts are affected.
      e.g. "host* router[0..3]"

    \item \ttt{@source}
      Optional parameter attribute that specifies the address of the source
      network. The default value is "*" that matches all sources.

    \item \ttt{@netmask}
      Optional parameter attribute that specifies the netmask of the source
      network. The default value is "*" that matches all sources.

    \item \ttt{@groups}
      Optional List of IPv4 multicast addresses specifying the groups this entry
      applies to. The default value is "*" that matches all multicast groups.
      e.g. "225.0.0.1 225.0.1.2".

    \item \ttt{@metric}
      Optional parameter attribute that specifies the metric in the route.

    \item \ttt{@parent}
      Optional parameter attribute that specifies the name of the interface
      the multicast datagrams are expected to arrive. When a datagram arrives
      on the parent interface, it will be forwarded towards the child interfaces;
      otherwise it will be dropped. The default value is the interface on the
      shortest path towards the source of the datagram.

    \item \ttt{@children}
      Mandatory parameter attribute that specifies a list of interface name
      patterns:
      \begin{compactitem}
        \item a name pattern (e.g. "ppp*") matches the name of the interface
        \item a 'towards' pattern (starting with ">", e.g. ">router*") matches the interface
         by naming one of the neighbour nodes on its link.
      \end{compactitem}
      Incoming multicast datagrams are forwarded to each child interface except the
      one they arrived in.
\end{compactitem}

The following example adds an entry to the multicast routing table of \ttt{router1},
that intsructs the routing algorithm to forward multicast datagrams whose source
is in the 10.0.1.0 network and whose destinatation address is 225.0.0.1 to
send on the \ttt{eth1} and \ttt{eth2} interfaces assuming it arrived on the
\ttt{eth0} interface:

\begin{verbatim}
<multicast-route hosts="router1" source="10.0.1.0" netmask="255.255.255.0"
                 groups="225.0.0.1" metric="10"
                 parent="eth0" children="eth1 eth2"/>
\end{verbatim}

\subsubsection*{Automatic route configuration}

If the \fpar{addStaticRoutes} parameter is true, then
the configurator add static routes to all routing tables.

The configurator uses Dijkstra's weighted shortest path algorithm to find
the desired routes between all possible node pairs. The resulting
routing tables will have one entry for all destination interfaces in the
network.

%     Weights will be infinite for IP nodes that have IP forwarding disabled
%     (to prevent routes from transiting them), and zero for all other nodes
%     (routers and and L2 devices). Edge weights are chosen to be inversely
%     proportional to the bitrate of the link, so that the configurator
%     prefers connections with higher bandwidth. For internal purposes,

The configurator can be safely instructed to add default routes
where applicable, significantly reducing the size of the host routing
tables. It can also add subnet routes instead of interface routes further
reducing the size of routing tables. Turning on this option requires
careful design to avoid having IP addresses from the same subnet on
different links.


\begin{caution}
Using manual routes and static route generation
together may have unwanted side effects, because route generation ignores
manual routes. Therefore if the configuration file contains
manual routes, then the \fpar{addStaticRoutes} parameter should be set
to \fkeyword{false}.
\end{caution}

\subsubsection*{Route optimization}

If the \fpar{optimizeRoutes} parameter is \fkeyword{true} then the
configurator tries to optimize the routing table for size.
This optimization allows configuring larger networks with smaller
memory footprint and makes the routing table lookup faster.

The optimization is performed by merging routes whose gateway and
outgoing interface is the same by finding a common prefix that
matches only those routes. The resulting routing table might be
different in that it will route packets that the original routing table
did not. Nevertheless the following invariant holds: any packet routed
by the original routing table (has matching route) will still be routed
the same way by the optimized routing table.

\subsubsection*{Parameters}

This list summarize the parameters of the \nedtype{IPv4NetorkConfigurator}:

\begin{params}
  \param{config}
   {XML configuration parameters for IP address assignment and adding manual routes.}
  \param{assignAddresses}
   {assign IP addresses to all interfaces in the network}
  \param{assignDisjunctSubnetAddresses}
   {avoid using the same address prefix and
    netmask on different links when assigning IP addresses to interfaces}
  \param{addStaticRoutes}
   {add static routes to the routing tables of all nodes
    to route to all destination interfaces (only where applicable; turn off when
    config file contains manual routes)}
  \param{addDefaultRoutes}
    {add default routes if all routes from a source node go
     through the same gateway (used only if addStaticRoutes is true)}
  \param{addSubnetRoutes}
   {add subnet routes instead of destination interface routes
    (only where applicable; used only if addStaticRoutes is true)}
  \param{optimizeRoutes}
   {optimize routing tables by merging routes, the resulting routing table might
    route more packets than the original (used only if addStaticRoutes is true)}
  \param{dumpTopology}
   {if true, then the module prints extracted network topology}
  \param{dumpAddresses}
   {if true, then the module prints assigned IP addresses for all interfaces}
  \param{dumpRoutes}
   {if true, then the module prints configured and optimized routing tables for all nodes to
    the module output}
  \param{dumpConfig}
   {name of the file, write configuration into the given config file that can be fed back
    to speed up subsequent runs (network configurations)}
\end{params}

\subsection{Ipv4FlatNetworkConfigurator (Legacy)}

The \nedtype{Ipv4FlatNetworkConfigurator} module configures
IP addresses and routes of IP nodes of a network.
All assigned addresses share a common subnet prefix,
the network topology will be ignored. Shortest path
routes are also generated from any node to any other
node of the network. The Gateway (next hop) field of the routes
is not filled in by these configurator, so it relies
on proxy ARP if the network spans several LANs.
It does not perform routing table optimization (i.e.
merging similar routes into a single, more general route.)

\begin{warning}
\nedtype{Ipv4FlatNetworkConfigurator} is considered
legacy, use do not use it for new projects.
\end{warning}

The \nedtype{Ipv4FlatNetworkConfigurator} module configures
the network when it is initialized. The configuration
is performed in stage 2, after interface tables are
filled in. Do not use a \nedtype{Ipv4FlatNetworkConfigurator}
module together with static routing files, because they
can iterfere with the configurator.

The \nedtype{Ipv4FlatNetworkConfigurator} searches each IP nodes of the network.
(IP nodes are those modules that have the @node NED property and
has a \nedtype{Ipv4RoutingTable} submodule named ``routingTable'').
The configurator then assigns IP addresses to the IP nodes, controlled
by the following module parameters:
\begin{itemize}
  \item \fpar{netmask} common netmask of the addresses (default is 255.255.0.0)
  \item \fpar{networkAddress} higher bits are the network part of the addresses,
        lower bits should be 0. (default is 192.168.0.0)
\end{itemize}

With the default parameters the assigned addresses are in the range
192.168.0.1 - 192.168.255.254, so there can be maximum 65534 nodes in the
network. The same IP address will be assigned to each interface
of the node, except the loopback interface which always has address 127.0.0.1
(with 255.0.0.0 mask).

After assigning the IP addresses, the configurator fills in the routing tables.
There are two kind of routes:
\begin{itemize}
  \item default routes: for nodes that has only one non-loopback interface
        a route is added that matches with any destination address
        (the entry has 0.0.0.0 \ttt{host} and \ttt{netmask} fields).
        These are remote routes, but the gateway address is left unspecified.
        The delivery of the datagrams rely on the proxy ARP feature of the
        routers.
  \item direct routes following the shortest paths: for nodes that has more
        than one non-loopback interface a separate route is added to each
        IP node of the network. The outgoing interface is chosen by the
        shortest path to the target node. These routes are
        added as direct routes, even if there is no direct link with the
        destination. In this case proxy ARP is needed to deliver the datagrams.
\end{itemize}

\begin{note}
This configurator does not try to optimize the routing tables.
If the network contains $n$ nodes, the size of all routing tables
will be proportional to $n^2$, and the time of the lookup of the
best matching route will be proportional to $n$.
\end{note}

% FIXME weird FlatNetworkConfigurator behaviour.
%       Assigned IP addresses does not mirror the hierachy of networks (e.g. each node in an Ethernet LAN handled as a one-element subnet).
%       No gateway address is set in the routes, delivery relies on proxy ARPing.
%       Direct routes created to each node, even if there is no direct link to it.
%       Different interfaces of a node should have different IP address.
%       Broadcast capable interfaces should have a real netmast (not 255.255.255.255) to support subnet directed IP broadcasts.

\subsection{Routing Files (Legacy)}
\label{subsec:routing_files}

Routing files are files with \ttt{.irt} or \ttt{.mrt} extension,
and their names are passed in the \fpar{routingFile} parameter
to \nedtype{Ipv4RoutingTable} modules.

Routing files may contain network interface configuration and static
routes. Both are optional. Network interface entries in the file
configure existing interfaces; static routes are added to the route table.

\begin{warning}
\nedtype{Routing files} are considered legacy, use do not use them for new
projects. Their contents can be expressed in \nedtype{Ipv4NetworkConfigurator}
config files.
\end{warning}

Interfaces themselves are represented in the simulation by modules
(such as the PPP module). Modules automatically register themselves
with appropriate defaults in the IPv4RoutingTable, and entries in the
routing file refine (overwrite) these settings.
Interfaces are identified by names (e.g. ppp0, ppp1, eth0) which
are normally derived from the module's name: a module called
\ttt{"ppp[2]"} in the NED file registers itself as interface ppp2.

An example routing file (copied here from one of the example simulations):

\begin{verbatim}
ifconfig:

# ethernet card 0 to router
name: eth0   inet_addr: 172.0.0.3   MTU: 1500   Metric: 1  BROADCAST MULTICAST
Groups: 225.0.0.1:225.0.1.2:225.0.2.1

# Point to Point link 1 to Host 1
name: ppp0   inet_addr: 172.0.0.4   MTU: 576   Metric: 1

ifconfigend.

route:
172.0.0.2   *           255.255.255.255  H  0   ppp0
172.0.0.4   *           255.255.255.255  H  0   ppp0
default:    10.0.0.13   0.0.0.0          G  0   eth0

225.0.0.1   *           255.255.255.255  H  0   ppp0
225.0.1.2   *           255.255.255.255  H  0   ppp0
225.0.2.1   *           255.255.255.255  H  0   ppp0

225.0.0.0   10.0.0.13   255.0.0.0        G  0   eth0

routeend.
\end{verbatim}

The \ttt{ifconfig...ifconfigend.} part configures interfaces,
and \ttt{route..routeend.} part contains static routes.
The format of these sections roughly corresponds to the output
of the \ttt{ifconfig} and \ttt{netstat -rn} Unix commands.

An interface entry begins with a \ttt{name:} field, and lasts until
the next \ttt{name:} (or until \ttt{ifconfigend.}). It may
be broken into several lines.

Accepted interface fields are:

\begin{itemize}
  \item \ttt{name:} - arbitrary interface name (e.g. eth0, ppp0)
  \item \ttt{inet\_addr:} - IP address
  \item \ttt{Mask:} - netmask
  \item \ttt{Groups:} Multicast groups. 224.0.0.1 is added automatically,
     and 224.0.0.2 also if the node is a router (IPForward==true).
  \item \ttt{MTU:} - MTU on the link (e.g. Ethernet: 1500)
  \item \ttt{Metric:} - integer route metric
  \item flags: \ttt{BROADCAST}, \ttt{MULTICAST}, \ttt{POINTTOPOINT}
\end{itemize}

The following fields are parsed but ignored: \ttt{Bcast},\ttt{encap},
\ttt{HWaddr}.

Interface modules set a good default for MTU, Metric (as $2*10^9$/bitrate) and
flags, but leave \fvar{inet\_addr} and \fvar{Mask} empty. \fvar{inet\_addr} and
\fvar{mask} should be set either from the routing file or by a dynamic network
configuration module.

The route fields are:

\begin{verbatim}
Destination  Gateway  Netmask  Flags  Metric Interface
\end{verbatim}

\fvar{Destination}, \fvar{Gateway} and \fvar{Netmask} have the usual meaning.
The \fvar{Destination} field should either be an IP address or ``default''
(to designate the default route). For \fvar{Gateway}, \ttt{*} is also
accepted with the meaning \ttt{0.0.0.0}.

\fvar{Flags} denotes route type:

\begin{itemize}
  \item \textit{H} ``host'': direct route (directly attached to the router), and
  \item \textit{G} ``gateway'': remote route (reached through another router)
\end{itemize}

\fvar{Interface} is the interface name, e.g. \ttt{eth0}.

\begin{important}
The meaning of the routes where the destination is a multicast address
has been changed in version 1.99.4. Earlier these entries was used
both to select the outgoing interfaces of multicast datagrams
sent by the higher layer (if multicast interface was otherwise unspecified)
and to select the outgoing interfaces of datagrams that are received from
the network and forwarded by the node.

From version 1.99.4 multicast routing applies reverse path forwarding.
This requires a separate routing table, that can not be populated from
the old routing table entries. Therefore simulations that use multicast
forwarding can not use the old configuration files, they should be
migrated to use an \nedtype{Ipv4NetworkConfigurator} instead.

Some change is needed in models that use link-local multicast too.
Earlier if the IP module received a datagram from the higher layer
and multiple routes was given for the multicast group,
then IP sent a copy of the datagram on each interface of that routes.
From version 1.99.4, only the first matching interface is used (considering
longest match). If the application wants to send the multicast datagram
on each interface, then it must explicitly loop and specify the multicast
interface.
\end{important}

% FIXME 'H' and 'G' flags should be independent. Now they excludes each other, the parser sets route.type to the last one.
%       H = host/network
%       G = indirect/direct

% TODO warn that multicast configuration has changed

\section{Configuring Layer 2}

The \nedtype{L2Configurator} module allows configuring network scenarios at layer 2.
The STP/RTP-related parameters such as link cost, port priority
and the ``is-edge'' flag can be configured with XML files.

This module is similar to \nedtype{Ipv4NetworkConfigurator}. It supports
the selector attributes \ttt{@hosts}, \ttt{@names}, \ttt{@towards}, \ttt{@among},
and they behave similarly to its \nedtype{Ipv4NetworkConfigurator} equivalent.
The \ttt{@ports} selector is also supported, for configuring per-port parameters.

The following example configures port 5 (if it exists) on all switches,
and sets cost=19 and priority=32768:

\begin{XML}
<config>
  <interface hosts='**' ports='5' cost='19' priority='32768'/>
</config>
\end{XML}

For more information about the usage of the selector attributes see
\nedtype{Ipv4NetworkConfigurator}.



%%% Local Variables:
%%% mode: latex
%%% TeX-master: "usman"
%%% End:


\cleardoublepage

\include{ch-lifecycle}
\cleardoublepage

\chapter{Scenario Management}
\label{cha:scenario-management}

TODO explain how INET supports custom simulation scenarios: creating new 
network nodes, shutting down network nodes, etc.

%%% Local Variables:
%%% mode: latex
%%% TeX-master: "usman"
%%% End:

\cleardoublepage

\include{ch-collecting-results}
\cleardoublepage

\chapter{Visualization}
\label{cha:visualization}

\section{Overview}

The INET Framework is able to visualize a wide range of events and conditions
in the network: packet drops, data link connectivity, wireless signal path loss,
transport connections, routing table routes, and many more. Visualization is
implemented as a collection of configurable INET modules that can be added
to simulations at will.

\section{Visualizing Network Communication}

\subsection{Visualizing Packet Drops}

Several network problems manifest themselves as excessive packet drops, for
example poor connectivity, congestion, or misconfiguration. Visualizing packet
drops helps identifying such problems in simulations, thereby reducing time
spent on debugging and analysis. Poor connectivity in a wireless network can
cause senders to drop unacknowledged packets after the retry limit is exceeded.
Congestion can cause queues to overflow in a bottleneck router, again resulting
in packet drops.

Packet drops can be visualized by including a \nedtype{PacketDropVisualizer}
module in the simulation. The \nedtype{PacketDropVisualizer} module indicates
packet drops by displaying an animation effect at the node where the packet drop
occurs. In the animation, a packet icon gets thrown out from the node icon, and
fades away.

The visualization of packet drops can be enabled with the visualizer's
\fpar{displayPacketDrops} parameter. By default, packet drops at all nodes are
visualized. This selection can be narrowed with the \fpar{nodeFilter},
\fpar{interfaceFilter} and \fpar{packetFilter} parameters.

One can click on the packet drop icon to display information about the packet
drop in the inspector panel.

Packets are dropped for the following reasons:

\begin{itemize}
  \item queue overflow
  \item retry limit exceeded
  \item unroutable packet
  \item network address resolution failed
  \item interface down
\end{itemize}


\subsection{Visualizing Transport Path Activity}

With INET simulations, it is often useful to be able to visualize network
traffic. INET provides several visualizers for this task, operating at various
levels of the network stack.

Transport path activity can be visualized by including a
\nedtype{TransportRouteVisualizer} module in the simulation. \nedtype{TransportRouteVisualizer}
that can provide graphical feedback about transport traffic, i.e. traffic that passes
through the transport layers of two endpoints. Adding an \nedtype{IntegratedVisualizer} is
also an option, because it also contains a \nedtype{TransportRouteVisualizer}. Transport
path activity visualization is disabled by default, it can be enabled by setting
the visualizer's \fpar{displayRoutes} parameter to true.

\nedtype{TransportRouteVisualizer} observes packets that pass through the transport layer,
i.e. carry data from/to higher layers.

The activity between two nodes is represented visually by a polyline arrow which
points from the source node to the destination node. \nedtype{TransportRouteVisualizer}
follows packets throughout their path so that the polyline goes through all
nodes which are the part of the path of packets. The arrow appears after the
first packet has been received, then gradually fades out unless it is reinforced
by further packets. Color, fading time and other graphical properties can be
changed with parameters of the visualizer.

By default, all packets and nodes are considered for the visualization. This
selection can be narrowed with the visualizer's packetFilter and nodeFilter
parameters.

\subsection{Visualizing Network Path Activity}

With INET simulations, it is often useful to be able to visualize network
traffic. INET offers several visualizers for this task, operating at various
levels of the network stack. In this showcase, we examine \nedtype{NetworkRouteVisualizer}
that can provide graphical feedback about network layer level traffic.

Network path activity can be visualized by including a \nedtype{NetworkRouteVisualizer}
module in the simulation. Adding an \nedtype{IntegratedVisualizer} module is also an
option, because it also contains a \nedtype{NetworkRouteVisualizer} module. Network path
activity visualization is disabled by default, it can be enabled by setting the
visualizer's \fpar{displayRoutes} parameter to true.

\nedtype{NetworkRouteVisualizer} currently observes packets that pass through the network
layer (i.e. carry data from/to higher layers), but not those that are internal
to the operation of the network layer protocol. That is, packets such as ARP,
although potentially useful, will not trigger the visualization.

The activity between two nodes is represented visually by a polyline arrow which
points from the source node to the destination node. \nedtype{NetworkRouteVisualizer}
follows packet throughout its path so the polyline goes through all nodes that
are part of the packet's path. The arrow appears after the first packet has been
received, then gradually fades out unless it is reinforced by further packets.
Color, fading time and other graphical properties can be changed with parameters
of the visualizer.

By default, all packets and nodes are considered for the visualization. This
selection can be narrowed with the visualizer's packetFilter and nodeFilter
parameters.


\subsection{Visualizing Data Link Activity}

With INET simulations, it is often useful to be able to visualize network
traffic. INET offers several visualizers for this task, operating at various
levels of the network stack. In this showcase, we examine \nedtype{DataLinkVisualizer}
that can provide graphical feedback about data link level traffic.

Data link activity can be visualized by including a \nedtype{DataLinkVisualizer} module in
the simulation. Adding an \nedtype{IntegratedVisualizer} module is also an option, because
it also contains a \nedtype{DataLinkVisualizer} module. Data link visualization is
disabled by default, it can be enabled by setting the visualizer's displayLinks
parameter to true.

\nedtype{DataLinkVisualizer} currently observes packets that pass through the data link
layer (i.e. carry data from/to higher layers), but not those that are internal
to the operation of the data link layer protocol. That is, frames such as ACK,
RTS/CTS, Beacon or Authentication/Association frames of IEEE 802.11, although
potentially useful, will not trigger the visualization. Visualizing such frames
may be implemented in future INET revisions.

The activity between two nodes is represented visually by an arrow that points
from the sender node to the receiver node. The arrow appears after the first
packet has been received, then gradually fades out unless it is refreshed by
further packets. The style, color, fading time and other graphical properties
can be changed with parameters of the visualizer.

By default, all packets, interfaces and nodes are considered for the
visualization. This selection can be narrowed to certain packets and/or nodes
with the visualizer's \fpar{packetFilter}, \fpar{interfaceFilter}, and
\fpar{nodeFilter} parameters.


\subsection{Visualizing Physical Link Activity}

With INET simulations, it is often useful to be able to visualize network
traffic. For this task, there are several visualizers in INET, operating at
various levels of the network stack. In this showcase, we demonstrate working of
\nedtype{PhysicalLinkVisualizer} that can provide graphical feedback about physical layer
traffic.

Physical link activity can be visualized by including a \nedtype{PhysicalLinkVisualizer}
module in the simulation. Adding an \nedtype{IntegratedVisualizer} module is also an
option, because it also contains a \nedtype{PhysicalLinkVisualizer} module. Physical link
activity visualization is disabled by default, it can be enabled by setting the
visualizer's \fpar{displayLinks} parameter to true.

\nedtype{PhysicalLinkVisualizer} observes frames that pass through the physical layer,
i.e. are received correctly.

The activity between two nodes is represented visually by a dotted arrow which
points from the sender node to the receiver node. The arrow appears after the
first frame has been received, then gradually fades out unless it is refreshed
by further frames. Color, fading time and other graphical properties can be
changed with parameters of the visualizer.

By default, all packets, interfaces and nodes are considered for the
visualization. This selection can be narrowed with the visualizer's
\fpar{packetFilter}, \fpar{interfaceFilter}, and \fpar{nodeFilter} parameters.

\ifdraft
\subsection{Visualizing the Transmission Medium}

TODO revise!

In order to help understanding the communication in the network the physical
layer supports visualizing its state. The following list shows what can be
displayed:

\begin{itemize}
  \item ongoing transmissions
  \item recent successful receptions
  \item recent obstacle intersections and surface normal vectors
\end{itemize}

The ongoing transmissions can be displayed with 3 dimensional spheres or with 2
dimensional rings laying in the XY plane. As the signal propagates through space
the figure grows with it to show where the beginning of the signal is. The inner
circle of the ring figure shows as the end of the signal propagates through
space. 

The recent successful receptions are displayed as straight lines between the
original positions of the transmission and the reception. The recent obstacle
intersections are also displayed as straight lines from the start of the
intersection to the end of it.
\fi

\subsection{Visualizing Routing Tables}

In a complex network topology, it is difficult to see how a packet would be
routed because the relevant data is scattered among network nodes and hidden in
their routing tables. INET contains support for visualization of routing tables,
and can display routing information graphically in a concise way. Using
visualization, it is often possible to understand routing in a simulation
without looking into individual routing tables. The visualization currently
supports IPv4.

The \nedtype{RoutingTableVisualizer} module (included in the network as part of
\nedtype{IntegratedVisualizer}) is responsible for visualizing routing table entries.

The visualizer basically annotates network links with labeled arrows that
connect source nodes to next hop nodes. The module visualizes those routing
table entries that participate in the routing of a given set of destination
addresses, by default the addresses of all interfaces of all nodes in the
network. That is, it selects the best (longest prefix) matching routes for all
destination addresses from each routing table, and shows them as arrows that
point to the next hop. Note that one arrow might need to represent several
routing entries, for example when distinct prefixes are routed towards the same
next hop.

Routing table entries are represented visually by solid arrows. An arrow going
from a source node represents a routing table entry in the source node's routing
table. The endpoint node of the arrow is the next hop in the visualized routing
table entry. By default, the routing entry is displayed on the arrows in
following format:

\begin{verbatim}
destination/mask -> gateway (interface)
\end{verbatim}

The format can be changed by setting the visualizer's \fpar{labelFormat} parameter.

Filtering is also possible. The \fpar{nodeFilter} parameter controls which nodes'
routing tables should be visualized (by default, all nodes), and the
\fpar{destinationFilter} parameter selects the set of destination nodes to consider
(again, by default all nodes.)

The visualizer reacts to changes. For example, when a routing protocol changes a
routing entry, or an IP address gets assigned to an interface by DHCP, the
visualizer automatically updates the visualizations according to the specified
filters. This is very useful e.g. for the simulation of mobile ad-hoc networks.

\subsection{Displaying IP Addresses and Other Interface Information}

In the simulation of complex networks, it is often useful to be able to display
node IP addresses, interface names, etc. above the node icons or on the links.
For example, when automatic address assignment is used in a hierarchical network
(e.g. using \nedtype{Ipv4NetworkConfigurator}), visual inspection can help to
verify that the result matches the expectations. While it is true that addresses and other
interface data can also be accessed in the GUI by diving into the interface
tables of each node, that is tedious, and unsuitable for getting an overview.

The \nedtype{InterfaceTableVisualizer} module (included in the network as part of
\nedtype{IntegratedVisualizer}) displays data about network nodes' interfaces.
(Interfaces are contained in interface tables, hence the name.) By default, the
visualization is turned off. When it is enabled using the
\fpar{displayInterfaceTables} parameter, the default is that interface names, IP
addresses and netmask length are displayed, above the nodes (for wireless
interfaces) and on the links (for wired interfaces). By clicking on an interface
label, details are displayed in the inspector panel.

The visualizer has several configuration parameters. The \fpar{format} parameter
specifies what information is displayed about interfaces. It takes a format
string, which can contain the following directives:

\begin{itemize}
  \item \%N: interface name
  \item \%4: IPv4 address
  \item \%6: IPv6 address
  \item \%n: network address. This is either the IPv4 or the IPv6 address
  \item \%l: netmask length
  \item \%M: MAC address
  \item \%\textbackslash: conditional newline for wired interfaces. The '\textbackslash'
  needs to be escaped with another '\textbackslash', i.e. '\%\textbackslash\textbackslash'
  \item \%i and \%s: the info() and str() functions for the interfaceEntry class, respectively
\end{itemize}

The default format string is
\texttt{"\%N \%\textbackslash\textbackslash\%n/\%l"}, i.e. interface name, IP address and
netmask length.

The set of visualized interfaces can be selected with the configurator's
\fpar{nodeFilter} and \fpar{interfaceFilter} parameters. By default, all
interfaces of all nodes are visualized, except for loopback addresses (the default for the
\fpar{interfaceFilter} parameter is \texttt{"not lo\textbackslash*"}.)

It is possible to display the labels for wired interfaces above the node icons,
instead of on the links. This can be done by setting the
\fpar{displayWiredInterfacesAtConnections} parameter to false.

There are also several parameters for styling, such as color and font selection.


\subsection{Visualizing IEEE 802.11 Network Membership}

When simulating wifi networks that overlap in space, it is difficult to see
which node is a member of which network. The membership may even change over
time. It would be useful to be able to display e.g. the SSID above node icons.

IEEE 802.11 network membership can be visualized by including a
\nedtype{Ieee80211Visualizer} module in the simulation. Adding an \nedtype{IntegratedVisualizer} is
also an option, because it also contains a \nedtype{Ieee80211Visualizer}. Displaying
network membership is disabled by default, it can be enabled by setting the
visualizer's \fpar{displayAssociations} parameter to true.

The \nedtype{Ieee80211Visualizer} displays an icon and the SSID above network nodes which
are part of a wifi network. The icons are color-coded according to the SSID. The
icon, colors, and other visual properties can be configured via parameters of
the visualizer.

The visualizer's \fpar{nodeFilter} parameter selects which nodes' memberships are
visualized. The \fpar{interfaceFilter} parameter selects which interfaces are
considered in the visualization. By default, all interfaces of all nodes are
considered.


\subsection{Visualizing Transport Connections}

In a large network with a complex topology, there might be many transport layer
applications and many nodes communicating. In such a case, it might be difficult
to see which nodes communicate with which, or if there is any communication at
all. Transport connection visualization makes it easy to get information about
the active transport connections in the network at a glance. Visualization makes
it easy to identify connections by their two endpoints, and to tell different
connections apart. It also gives a quick overview about the number of
connections in individual nodes and the whole network.

The \nedtype{TransportConnectionVisualizer} module (also part of \nedtype{IntegratedVisualizer})
displays color-coded icons above the two endpoints of an active, established
transport layer level connection. The icons will appear when the connection is
established, and disappear when it is closed. Naturally, there can be multiple
connections open at a node, thus there can be multiple icons. Icons have the
same color at both ends of the connection. In addition to colors, letter codes
(A, B, AA, …) may also be displayed to help in identifying connections. Note
that this visualizer does not display the paths the packets take. If you are
interested in that, take a look at \nedtype{TransportRouteVisualizer}, covered in the
Visualizing Transport Path Activity showcase.

The visualization is turned off by default, it can be turned on by setting the
\fpar{displayTransportConnections} parameter of the visualizer to true.

It is possible to filter the connections being visualized. By default, all
connections are included. Filtering by hosts and port numbers can be achieved by
setting the \fpar{sourcePortFilter}, \fpar{destinationPortFilter},
\fpar{sourceNodeFilter} and \fpar{destinationNodeFilter} parameters.

The icon, colors and other visual properties can be configured by setting the
visualizer's parameters.


\section{Visualizing The Infrastructure}

\subsection{Visualizing the Physical Environment}

The physical environment has a profound effect on the communication of wireless
devices. For example, physical objects like walls inside buildings constraint
mobility. They also obstruct radio signals often resulting in packet loss. It's
difficult to make sense of the simulation without actually seeing where physical
objects are.

The visualization of physical objects present in the physical environment is
essential.

The \nedtype{PhysicalEnvironmentVisualizer} (also part of \nedtype{IntegratedVisualizer}) is
responsible for displaying the physical objects. The objects themselves are
provided by the PhysicalEnvironment module; their geometry, physical and visual
properties are defined in the XML configuration of the PhysicalEnvironment
module.

The two-dimensional projection of physical objects is determined by the
\nedtype{SceneCanvasVisualizer} module. (This is because the projection is also needed by
other visualizers, for example \nedtype{MobilityVisualizer}.) The default view is top view
(z axis), but you can also configure side view (x and y axes), or isometric or
ortographic projection.

The visualizer also supports OpenGL-based 3D rendering using the OpenSceneGraph
(OSG) library. If the OMNeT++ installation has been compiled with OSG
support, you can switch to 3D view using the Qtenv toolbar.

\subsection{Visualizing Node Mobility}

In INET simulations, the movement of mobile nodes is often as important as the
communication among them. However, as mobile nodes roam, it is often difficult
to visually follow their movement. INET provides a visualizer that not only
makes visually tracking mobile nodes easier, but also indicates other properties
like speed and direction.

Node mobility of nodes can be visualized by \nedtype{MobilityVisualizer} module
(included in the network as part of \nedtype{IntegratedVisualizer}). By default,
mobility visualization is enabled, it can be disabled by setting
\fpar{displayMovements} parameter to false.

By default, all mobilities are considered for the visualization. This selection
can be narrowed with the visualizer's \fpar{moduleFilter} parameter.

The visualizer has several important features:

\begin{itemize}
  \item Movement Trail: It displays a line along the recent path of movements.
        The trail gradually fades out as time passes. Color, trail length and
        other graphical properties can be changed with parameters of the
        visualizer.
  \item Velocity Vector: Velocity is represented visually by an arrow. Its
        starting point is the node, and its direction coincides with the
        movement's direction. The arrow's length is proportional to the node's
       speed.
  \item Orientation Arc: Node orientation is represented by an arc whose size
       is specified by the \fpar{orientationArcSize} parameter. This value is the
       relative size of the arc compared to a full circle. The arc's default
       value is 0.25, i.e. a quarter of a circle.
\end{itemize}

These features are disabled by default; they can be enabled by setting the
visualizer's \fpar{displayMovementTrails}, \fpar{displayVelocities} and
\fpar{displayOrientations} parameters to true.


\section{Generic}

The following pages describe generic features that are common to many visualizers.

TODO Styling and Appearance (from the showcase)


%%% Local Variables:
%%% mode: latex
%%% TeX-master: "usman"
%%% End:


\cleardoublepage

\include{ch-authors-guide}
\cleardoublepage

\include{ch-history}
\cleardoublepage

\bibliographystyle{alpha}
\bibliography{inet-users-guide}


%% no need for the following since 'tocbibind' package
%% \phantomsection
%% \addcontentsline{toc}{chapter}{\indexname}
\printindex

\end{document}

%%% Local Variables:
%%% mode: latex
%%% TeX-master: t
%%% End:
